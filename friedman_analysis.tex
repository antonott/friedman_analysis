\documentclass{report}
\usepackage[utf8]{inputenc}
\usepackage{amsmath}
\usepackage{amssymb}
\usepackage{amsthm}
\usepackage{mathrsfs}
\usepackage{enumitem}
\usepackage{array}

\newcommand{\bb}[1]{\mathbb{#1}}
\newcommand{\norm}[1]{\lVert #1 \rVert}
\newcommand{\brac}[1]{\langle #1 \rangle}

\let\sc\relax
\newcommand{\sc}[1]{\mathscr{#1}}

\let\cal\relax
\newcommand{\cal}[1]{\mathcal{#1}}

\theoremstyle{remark}
\newtheorem*{solution}{Solution}

\setenumerate{listparindent=\parindent, parsep=0pt}

\title{Notes for \textit{Foundations of Modern Analysis} by Avner Friedman}
\author{Anton Ottosson -- antonott@kth.se}
\date{\today}

\begin{document}

\maketitle

\chapter*{Chapter 1 -- Measure Theory}

\section*{Section 1.1 -- Rings and Algebras}

\subsection*{Problems}

\subsubsection*{1.1.1}
\begin{equation*}
  \left( \varliminf_n E_n \right)^c = \varlimsup_n E_n^c, \quad \left( \varlimsup_n E_n \right)^c = \varliminf_n E_n^c.
\end{equation*}

\begin{solution}
  Note that
  \begin{equation*}
    \begin{split}
      x \in \varliminf_n E_n &\iff \text{$x \in E_n$ for all but finitely many $n$} \\
      &\iff \text{$x \in E_n^c$ for only finitely many $n$}.
    \end{split}
  \end{equation*}
  Hence
  \begin{equation*}
    \begin{split}
      x \in \left( \varliminf_n E_n \right)^c &\iff \text{$x \in E_n^c$ for infinitely many $n$} \\
      &\iff x \in \varlimsup_n E_n^c,
    \end{split}
  \end{equation*}
  proving the first identity.

  Next, let $F_n = E_n^c$ for every $n$. Then
  \begin{equation*}
    \varlimsup_n E_n = \varlimsup_n F_n^c = \left( \varliminf_n F_n \right)^c = \left( \varliminf_n E_n^c \right)^c
  \end{equation*}
  by the first identity, and the second identity follows.
\end{solution}

\subsubsection{1.1.2}
\begin{equation*}
  \varlimsup_n E_n = \bigcap_{k=1}^\infty \bigcup_{n=k}^\infty E_n, \quad \varliminf_n E_n = \bigcup_{k=1}^\infty \bigcap_{n=k}^\infty E_n.
\end{equation*}

\begin{solution}
  Suppose $x \in \varlimsup_n E_n$. Then $x \in E_n$ for infinitely many $n$. It follows that $x \in \bigcup_{n=k}^\infty E_n$ for all $k \in \bb N$, and hence that $x \in \bigcap_{k=1}^\infty \bigcup_{n=k}^\infty E_n$.

  Conversely, assume that $x \in \bigcap_{k=1}^\infty \bigcup_{n=k}^\infty E_n$. Then $x \in \bigcup_{n=k}^\infty E_n$ for all $k \in \bb N$. It follows that $x \in E_n$ for infinitely many $n$, and thus that $x \in \varlimsup_n E_n$. This proves the first identity.

  Next, suppose that $x \in \varliminf_n E_n$. Then $x \in E_n$ for all but finitely many $n$, so there is some $k' \in \bb N$ such that $x \in E_n$ for all $n \ge k'$. It follows that $x \in \bigcap_{n=k'}^\infty E_n$, and hence that $x \in \bigcup_{k=1}^\infty \bigcap_{n=k}^\infty E_n$.

  Conversely, assume that $x \in \bigcup_{k=1}^\infty \bigcap_{n=k}^\infty E_n$. Then $x \in \bigcap_{n=k'}^\infty E_n$ for some $k' \in \bb N$, which means that $x \in E_n$ for all $n \ge k'$. It follows that $x \in E_n$ for all but finitely many $n$; that is, $x \in \varliminf_n E_n$.
\end{solution}

\subsubsection*{1.1.3}
If $\sc R$ is a $\sigma$-ring and $E_n \in \sc R$, then
\begin{equation*}
  \bigcap_{n=1}^\infty E_n \in \sc R, \quad \varlimsup_n E_n \in \sc R, \quad \varliminf_n E_n \in \sc R.
\end{equation*}

\begin{solution}
  Let $Y = \bigcup_{n=1}^\infty E_n \in \sc R$. Then $E_n \subset Y$ for all $Y$, and it follows that
  \begin{equation*}
    \bigcap_{n=1}^\infty E_n = Y \cap \left( \bigcap_{n=1}^\infty E_n \right) = Y - \left( Y - \bigcap_{n=1}^\infty E_n \right).
  \end{equation*}
  Notice that
  \begin{equation*}
    Y - \bigcap_{n=1}^\infty E_n = \bigcup_{n=1}^\infty (Y - E_n) \in \sc R,
  \end{equation*}
  by properties (b) and (e). (The equality is analogous to the identity (1.1.2), but with $Y$ in place of $X$.) It follows (by (b) again) that $\bigcap_{n=1}^\infty E_n \in \sc R$. For later reference, let us call this result (x).

  Given $k \in \bb N$, let $A_n = \varnothing$ for $n < k$, and let $A_n = E_n$ for $n \ge k$. Then $A_n \in \sc R$ for all $n$ by (a), hence
  \begin{equation*}
    \bigcup_{n=k}^\infty E_n = \bigcup_{n=1}^\infty A_n \in \sc R
  \end{equation*}
  by (e). It then follows by (x) that
  \begin{equation*}
    \varlimsup_n E_n = \bigcap_{k=1}^\infty \bigcup_{n=k}^\infty E_n \in \sc R.
  \end{equation*}

  By a similar argument we find that (x) implies
  \begin{equation*}
    \bigcap_{n=k}^\infty E_n \in \sc R
  \end{equation*}
  for all $k \in \bb N$. Thus
  \begin{equation*}
    \varliminf_n E_n = \bigcup_{k=1}^\infty \bigcap_{n=k}^\infty E_n \in \sc R
  \end{equation*}
  by (e).
\end{solution}

\subsubsection*{1.1.4}
The intersection of any collection of rings (algebras, $\sigma$-rings, or $\sigma$-algebras) is also a ring (an algebra, $\sigma$-ring, or $\sigma$-algebra).

\begin{solution}
  Let $\sc C$ be a collection of classes. Let $\bigcap \sc C$ denote the intersection of all classes in $\sc C$. We will show that if one of the properties (a)-(e) is satisfied by all classes in $\sc C$, then $\bigcap \sc C$ satisfies that property as well. The result requested in the problem then follows as an immediate corollary.

  It is clear that if every $\sc R \in \sc C$ satisfies (a), then so does $\bigcap \sc C$. Suppose every $\sc R \in \sc C$ satisfies (b). If $A,B \in \bigcap \sc C$ then $A,B \in \sc R$ for every $\sc R \in \sc C$. Hence $A - B \in \sc R$ for all $\sc R \in \sc C$, and it follows that $A - B \in \bigcap \sc C$. The argument for (c) is similar (with $A \cup B$ in place of $A - B$), and (d) is obvious.

  Finally, suppose that every $\sc R \in \sc C$ satisfies (e). If $A_1, A_2, \dotsc \in \bigcap \sc C$ then $A_1, A_2, \dotsc \in \sc R$ for every $\sc R \in \sc C$. Hence $\bigcup_{n=1}^\infty A_n \in \sc R$ for all $\sc R \in \sc C$, and it follows that $\bigcup_{n=1}^\infty A_n \in \bigcap \sc C$.
\end{solution}

\subsubsection*{1.1.5}
If $\sc D$ is any class of sets, then there exists a unique ring $\sc R_0$ such that (i) $\sc R_0 \supset \sc D$, and (ii) any ring $\sc R$ containing $\sc D$ contains also $\sc R_0$. $\sc R_0$ is called the \emph{ring generated} by $\sc D$, and is denoted by $\sc R(\sc D)$.

\begin{solution}
  Let $\sc R_0$ be the intersection of all rings containing $\sc D$. This is a ring by the previous exercise, and it satisfies the properties (i) and (ii). To see that it is unique, let $\sc R_0'$ also by a ring satisfying (i) and (ii). Then $\sc R_0 \subset \sc R_0'$ and $\sc R_0' \subset \sc R_0$ by property (ii), so $\sc R_0 = \sc R_0'$.
\end{solution}

\subsubsection*{1.1.6}
If $\sc D$ is any class of sets, then there exists a unique $\sigma$-ring $\sc S_0$ such that (i) $\sc S_0 \supset \sc D$, and (ii) any $\sigma$-ring containing $\sc D$ contains also $\sc S_0$. We call $\sc S_0$ the \emph{$\sigma$-ring generated} by $\sc D$, and denote it by $\sc S(\sc D)$. A similar result holds for $\sigma$-algebras, and we speak of the \emph{$\sigma$-algebra generated} by $\sc D$.

\begin{solution}
  By the same argument as in the previous exercise, $\sc S_0$ is the intersection of all $\sigma$-rings containing $\sc D$. Similarly the $\sigma$-algebra generated by $\sc D$ is the intersection of all $\sigma$-algebras containing $\sc D$.
\end{solution}

\subsubsection*{1.1.7}
If $\sc D$ is any class of sets, then every set in $\sc R(\sc D)$ can be covered by (that is, is contained in) a finite union of sets of $\sc D$. [\emph{Hint:} The class $\sc K$ of sets that can be covered by finite unions of sets of $\sc D$ forms a ring.]

\begin{solution}
  Let $\sc K$ be the class of all sets that can be covered by a finite union of sets in $\sc D$. Certainly $\varnothing \in \sc K$, since $\varnothing$ is a subset of the empty union. If $A, B \in \sc K$, then
  \begin{equation*}
    A \subset \bigcup_{i=1}^m E_i, \quad B \subset \bigcup_{i=1}^n F_i,
  \end{equation*}
  for some sets $E_1, \dots, E_m, F_1, \dots, F_n \in \sc D$. (Note that $m$ or $n$ can be zero, in which case the corresponding union is empty.) Thus
  \begin{equation*}
    A - B \subset A \subset \bigcup_{i=1}^m E_i
  \end{equation*}
  and
  \begin{equation*}
    A \cup B \subset \left( \bigcup_{i=1}^m E_i \right) \cup \left( \bigcup_{j=1}^n F_j \right),
  \end{equation*}
  so both $A - B$ and $A \cup B$ are elements of $\sc K$.

  The above shows that $\sc K$ is a ring, and certainly $\sc D \subset \sc K$. Hence $\sc R(\sc D) \subset \sc K$ by Problem 1.1.5, and it follows that every set in $\sc R(\sc D)$ can be covered by a finite union of sets in $\sc D$.
\end{solution}

\section*{Section 1.1 -- Definition of Measure}

\subsection*{Problems}

\subsubsection*{1.2.1}
If $\mu$ satisfies the properties (i)-(iii) in Definition 1.2.1, and if $\mu(E) < \infty$ for at least one set $E$, then (iv) is also satisfied.

\begin{solution}
  We have
  \begin{equation*}
    \mu(E) = \mu(E \cup \varnothing) = \mu(E) + \mu(\varnothing),
  \end{equation*}
  hence $\mu(\varnothing) = 0$.
\end{solution}

\subsubsection*{1.2.2}
Let $X$ be an infinite space. Let $\cal A$ be the class of all subsets of $X$. Define $\mu(E) = 0$ if $E$ is finite and $\mu(E) = \infty$ if $E$ is infinite. Then $\mu$ is finitely additive but not completely additive.

\begin{solution}
  Suppose $A, B \in \cal A$. Note that $A \cup B$ is finite if both $A$ and $B$ are finite, but infinite otherwise. Hence
  \begin{equation*}
    \mu(A \cup B) = 0 = \mu(A) + \mu(B)
  \end{equation*}
  in the former case, and
  \begin{equation*}
    \mu(A \cup B) = \infty = \mu(A) + \mu(B)
  \end{equation*}
  in the latter. This proves that $\mu$ is additive; \emph{finite} additivity follows by a simple induction argument.

  Let $(x_n)$ be a sequence of distinct points in $X$. Then $\bigcup_{n=1}^\infty \{x_n\}$ is an infinite set, so
  \begin{equation*}
    \mu \left( \bigcup_{n=1}^\infty \{x_n\} \right) = \infty,
  \end{equation*}
  but
  \begin{equation*}
    \sum_{n=1}^\infty \mu(\{x_n\}) = 0.
  \end{equation*}
  Thus $\mu$ is not completely additive.
\end{solution}

\subsubsection*{1.2.3}
If $\mu$ is a measure on a $\sigma$-algebra $\cal A$, and if $E$, $F$ are sets of $\cal A$, then
\begin{equation*}
  \mu(E) + \mu(F) = \mu(E \cup F) + \mu(E \cap F).
\end{equation*}

\begin{solution}
  If $\mu(F) = \infty$, then $\mu(E \cup F) = \infty$ by Theorem 1.2.1(i), and the given equality holds. If $\mu(F) < \infty$, then
  \begin{equation*}
    \begin{split}
      \mu(E \cup F) &= \mu[E \cup (F - (E \cap F))] \\
      &= \mu(E) + \mu[F - (E \cap F)] \\
      &= \mu(E) + \mu(F) - \mu(E \cap F),
    \end{split}
  \end{equation*}
  with the last equality following from Theorem 1.2.1(ii). Note that $E \cap F \subset F$ so that $\mu(E \cap F) \le \mu(F) < \infty$. Hence we can rearrange the above to yield
  \begin{equation*}
    \mu(E) + \mu(F) = \mu(E \cup F) + \mu(E \cap F).
  \end{equation*}
\end{solution}

\subsubsection*{1.2.6}
Give an example of a measure $\mu$ and a monotone-decreasing sequence $\{E_n\}$ of $\cal A$ such that $\mu(E_n) = \infty$ for all $n$, and $\mu(\lim_n E_n) = 0$.

\begin{solution}
  Let $X = \bb R$ and let $\cal A = \cal P(\bb R)$ (the power set of $\bb R$; this is easily seen to be a $\sigma$-algebra). Define $\mu: \cal A \to [0,\infty]$ such that $\mu(E)$ is the number of points in $E$ (with $\mu(E) = \infty$ if $E$ is infinite). This is easily seen to be a measure.

  For each $n \in \bb N$, let $E_n = (0, 1/n)$. Then $(E_n)$ is a monotone decreasing sequence of sets in $\cal A$, $\mu(E_n) = \infty$ for all $n$, and
  \begin{equation*}
    \mu \left( \lim_n E_n \right) = \mu \left( \bigcap_{n=1}^\infty E_n \right) = \mu(\varnothing) = 0.
  \end{equation*}
\end{solution}

\section*{Section 1.3 -- Outer Measure}

\subsection*{Problems}

\subsubsection*{1.3.1}
Define $\mu^*(E)$ as the number of points in $E$ if $E$ is finite and $\mu^*(E) = \infty$ if $E$ is infinite. Show that $\mu^*$ is an outer measure. Determine the measurable sets.

\begin{solution}
  Of the properties listed in Definition 1.3.1, only countable subadditivity is non-obvious for $\mu^*$. But let us start with proving finite subadditivity.

  Let $A$ and $B$ be sets. If either is infinite, then so is $A \cup B$, hence
  \begin{equation*}
    \mu^*(A \cup B) = \infty = \mu^*(A) + \mu^*(B).
  \end{equation*}
  If both $A$ and $B$ are finite sets, then
  \begin{equation*}
    \mu^*(A \cup B) = \mu^*(A) + \mu^*(B - A) \le \mu^*(A) + \mu^*(B)
  \end{equation*}
  by basic set-theoretic considerations. Thus $\mu^*$ is subadditive, and finite subadditivity follows by induction on the number of sets in the union.

  Now, let $(E_n)$ be a sequence of sets. If infinitely many of the sets $E_n$ are nonempty, then $\sum_n \mu^*(E_n) = \infty$, and
  \begin{equation*}
    \mu^*\left( \bigcup_n E_n \right) \le \sum_n \mu^*(E_n)
  \end{equation*}
  follows. If only finitely many of the sets $E_n$ are nonempty, let $E_{n_1}, E_{n_2}, \dots, E_{n_k}$ be those sets. Then
  \begin{equation*}
    \mu^*\left( \bigcup_{n=1}^\infty E_n \right) = \mu^*\left( \bigcup_{i=1}^k E_{n_i} \right) \le \sum_{i=1}^k \mu^*(E_{n_i}) = \sum_{n=1}^\infty \mu^*(E_n),
  \end{equation*}
  by finite subadditivity. This proves that $\mu^*$ is countably subadditive, and hence that $\mu^*$ is an outer measure.

  Note that $\mu^*$ is \emph{additive} on disjoint sets; if $A \cap B = \varnothing$, then $\mu^*(A \cup B) = \mu^*(A) + \mu^*(B)$. In particular,
  \begin{equation*}
    \mu^*(A) = \mu^*(A \cap E) + \mu^*(A - E)
  \end{equation*}
  for all sets $A, E$. That is, all sets are measurable.
\end{solution}

\subsubsection*{1.3.2}
Define $\mu^*(\varnothing) = 0$, $\mu^*(E) = 1$ if $E \ne \varnothing$. Show that $\mu^*$ is an outer measure, and determine the measurable sets.

\begin{solution}
  As in the previous excercise, the only slightly non-obvious property is countable subadditivity. Hence, let $(E_n)$ be a sequence of sets. If all the sets $E_n$ are empty, then certainly
  \begin{equation*}
    \mu^* \left( \bigcup_n E_n \right) = 0 = \sum_{n} \mu^*(E_n).
  \end{equation*}
  If not, then there is some $m$ such that $E_m \ne \varnothing$, and it follows that
  \begin{equation*}
    \mu^* \left( \bigcup_n E_n \right) = 1 = \mu^*(E_m) \le \sum_n \mu^*(E_n).
  \end{equation*}
  Thus $\mu^*$ is indeed countably subadditive, and therefore also an outer measure.

  The empty set is measurable:
  \begin{equation*}
    \mu^*(A \cap \varnothing) + \mu^*(A - \varnothing) = \mu^*(\varnothing) + \mu^*(A) = \mu^*(A)
  \end{equation*}
  for all sets $A$. It follows by Theorem 1.3.1 that $X$ is measurable as well (the measurable sets make up a $\sigma$-algebra). Indeed $\varnothing$ and $X$ are the only measurable sets. To see this, let $E$ be any set other than those two (this requires that $X$ contains at least two elements). Then both $E$ and $E^c$ are nonempty, so
  \begin{equation*}
    \mu^*(X \cap E) + \mu^*(X - E) = \mu^*(E) + \mu^*(E^c) = 2 > 1 = \mu^*(X).
  \end{equation*}
\end{solution}


\section*{Section 1.4 -- Construction of Outer Measures}

\subsection*{Problems}

\subsubsection*{1.4.4}
If $\sc K$ is a $\sigma$-algebra and $\lambda$ is a measure on $\sc K$, then $\mu^*(A) = \lambda(A)$ for any $A \in \sc K$. [\emph{Hint:} $\mu^*(A) = \inf \{\lambda(E); E \in \sc K, E \supset A\}$.]

\begin{solution}
  Note that the description of $\mu^*$ can be simplified when $\sc K$ is a $\sigma$-algebra and $\lambda$ is a measure. For suppose that $A \subset X$, $E_n \in \sc K \; (n = 1, 2, \dots$), and $A \subset \bigcup_n E_n$. Then $E := \bigcup_n E_n \in \sc K$, and $\lambda(E) \le \sum_n \lambda(E_n)$ by Theorem 1.2.2. Hence
  \begin{equation*}
    \mu^*(A) = \inf \{\lambda(E); E \in \sc K, E \supset A\}.
  \end{equation*}

  Now, suppose that $A \in \sc K$. Certainly $\lambda(A)$ is an element of $\{\lambda(E); E \in \sc K, E \supset A\}$. And if $E \in \sc K$ and $E \supset A$, then $\lambda(E) \ge \lambda(A)$ by Theorem 1.2.1(i). Thus
  \begin{equation*}
    \lambda(A) = \inf \{\lambda(E); E \in \sc K, E \supset A\} = \mu^*(A).
  \end{equation*}
\end{solution}

\subsubsection*{1.4.5}
If $\sc K$ is a $\sigma$-algebra and $\lambda$ is a measure on $\sc K$, then every set in $\sc K$ is $\mu^*$-measurable.

\begin{solution}
  Recall the simplified description of $\mu^*$ from the previous problem. Let $E \in \sc K$ and $A \subset X$. For every $\epsilon > 0$ there exists $F \in \sc K$ such that $F \supset A$ and
  \begin{equation*}
    \mu^*(A) + \epsilon > \lambda(F);
  \end{equation*}
  else $\mu^*(A)$ would not be the greatest lower bound of $\{\lambda(E); E \in \sc K, E \supset A\}$. Moreover,
  \begin{equation*}
    \lambda(F) = \lambda(F \cap E) + \lambda(F - E)
  \end{equation*}
  since $\lambda$ is a measure on $\sc K$,
  \begin{equation*}
    \lambda(F \cap E) + \lambda(F - E) = \mu^*(F \cap E) + \mu^*(F - E) 
  \end{equation*}
  by what we found in the previous exercise, and finally
  \begin{equation*}
    \mu^*(F \cap E) + \mu^*(F - E) \ge \mu^*(A \cap E) + \mu^*(A - E)
  \end{equation*}
  by monotonicity of the outer measure $\mu^*$. Putting all of this together, we have
  \begin{equation*}
    \mu^*(A) + \epsilon > \mu^*(A \cap E) + \mu^*(A - E)
  \end{equation*}
  for all $\epsilon > 0$, and thus
  \begin{equation*}
    \mu^*(A) \ge \mu^*(A \cap E) + \mu^*(A - E).
  \end{equation*}
  It follows that every set $E \in \sc K$ is $\mu^*$-measurable.
\end{solution}


\section*{Section 1.6 -- The Lebesgue and the Lebesgue-Stieltjes Measures}

\subsection*{Problems}

\subsubsection*{1.6.3}
The outer Lebesgue measure of a closed bounded interval $[a,b]$ on the real line is equal to $b-a$. [\emph{Hint:} Use the Heine-Borel theorem to replace a countable covering by a finite covering.]

\begin{solution}
  Suppose $(E_n)$ is a sequence of elements of $\sc K$ (i.e.\ a sequence of open intervals) such that $[a,b] \subset \bigcup_{n=1}^\infty E_n$. The collection $\{E_n\}$ constitutes an \emph{open cover} of $[a,b]$. By the Heine-Borel theorem $[a,b]$ is compact, hence there exists a \emph{finite subcover} $\{E_{n_1}, \dots, E_{n_k}\}$, such that $[a,b] \subset \bigcup_{i=1}^k E_{n_i}$.

  Assume without loss of generality that $E_{n_i} \cap [a,b] \ne \varnothing$ for all $i$; otherwise we can simply remove those $E_{n_i}$ that are disjoint with $[a,b]$ and still have a finite subcover. Write $E_{n_i} = (a_i, b_i)$ for each $i$, and define
  \begin{equation*}
    \alpha = \min\{a_1, \dots, a_k\}, \quad \beta = \max\{b_1, \dots, b_k\}.
  \end{equation*}
  It is clear that $\alpha$ and $\beta$ are the infimum and supremum, respectively, of $\bigcup_{i=1}^k E_{n_i}$. Note that $\alpha = a_j$ for some $j$, and $a_j < a < b_j$ since $E_{n_j}$ and $[a,b]$ have nonempty intersection. Thus $(\alpha, a] \subset \bigcup_{i=1}^k E_{n_i}$, and similarly $[b, \beta) \subset \bigcup_{i=1}^k E_{n_i}$. It follows that
  \begin{equation*}
    \bigcup_{i=1}^k E_{n_i} = (\alpha, \beta) \in \sc K.
  \end{equation*}

  Finally note that $\lambda$ is finitely subadditive. (This is easily proven with induction.) (TODO: This is not convincing; use better proof from Rosenthal notes.) Thus,
  \begin{equation*}
    \sum_{n=1}^\infty \lambda(E_n) \ge \sum_{i=1}^k \lambda(E_{n_i}) \ge \lambda\left[ (\alpha, \beta) \right] = \beta - \alpha > b - a.
  \end{equation*}
  It follows that $b - a$ is a lower bound of the set
  \begin{equation*}
    \Lambda([a,b]) := \left\{ \sum_{n=1}^\infty \lambda(E_n); \, E_n \in \sc K, \, \bigcup_{n=1}^\infty E_n \supset [a,b] \right\}.
  \end{equation*}
  Moreover, for every $\epsilon > 0$ we have
  \begin{equation*}
    [a,b] \subset \left( a - \frac{\epsilon}{2}, b + \frac{\epsilon}{2} \right) \in \sc K
  \end{equation*}
  and
  \begin{equation*}
    \lambda \left[ \left( a - \frac{\epsilon}{2}, b + \frac{\epsilon}{2} \right) \right] = b - a + \epsilon.
  \end{equation*}
  Hence $b - a$ is the \emph{greatest} lower bound of $\Lambda([a,b])$, and $\mu^*([a,b]) = b - a$.
\end{solution}

\subsubsection*{1.6.4}
The outer Lebesgue measure of each of the intervals $(a,b), [a,b), (a,b]$ is equal to $b-a$.

\begin{solution}
  Recall that $\mu^*$ is monotone, on account of being an outer measure. Hence $\mu^*[(a,b)] \le \mu^*([a,b]) = b - a$, the latter equality being the result of the previous problem. Moreover, for all $\epsilon \in (0, b-a)$ we have
  \begin{equation*}
    \left( a + \frac{\epsilon}{2}, b - \frac{\epsilon}{2} \right) \subset (a,b),
  \end{equation*}
  so that
  \begin{equation*}
    \mu^*[(a,b)] \ge \mu^* \left[ \left( a + \frac{\epsilon}{2}, b - \frac{\epsilon}{2} \right) \right] = b - a + \epsilon.
  \end{equation*}
  Thus $\mu^*[(a,b)] \ge b - a$, and it follows that $\mu^*[(a,b)] = b - a$.

  The outer measures of $[a,b)$ and $(a,b]$ follow immediately by monotonicity:
  \begin{equation*}
    \mu^*[(a,b)] \le \mu^*([a,b)) \le \mu^*([a,b]),
  \end{equation*}
  so that $\mu^*([a,b)) = b - a$. Similarly for $(a,b]$.
\end{solution}

\subsubsection*{1.6.5}
Consider the transformation $T x = \alpha x + \beta$ from the real line onto itself, where $\alpha, \beta$ are real numbers and $\alpha \ne 0$. It maps sets $E$ onto sets $T(E)$. Denote by $\mu$ ($\mu^*$) the Lebesgue measure (outer measure) on the real line. Prove
\begin{enumerate}[label=(\alph*)]
  \item For any set $E$, $\mu^*(T(E)) = |\alpha| \mu^*(E)$.
  \item $E$ is Lebesgue-measurable if and only if $T(E)$ is Lebesgue-measureable.
  \item If $E$ is Lebesgue-measurable, then $\mu(T(E)) = |\alpha| \mu(E)$.
\end{enumerate}

\begin{solution}
  Let us start with a couple of simple observations:
  \begin{itemize}
    \item $T$ is bijective, with inverse given by
      \begin{equation*}
        T^{-1}(x) = \frac{x - \beta}{\alpha}.
      \end{equation*}
    \item Suppose $I = (a,b)$. Then
      \begin{equation*}
        T(I) = (\alpha a + \beta, \alpha b + \beta)
      \end{equation*}
      if $\alpha > 0$, and
      \begin{equation*}
        T(I) = (\beta b + \beta, \alpha a + \beta)
      \end{equation*}
      if $\alpha < 0$. Either way,
      \begin{equation*}
        \mu^*[T(I)] = |\alpha| (b - a) = |\alpha| \mu^*(I),
      \end{equation*}
      where we have used one of the results of the previous exercise. Similarly, $T^{-1}(I)$ is an open interval and
      \begin{equation*}
        \mu^*[T^{-1}(I)] = |\alpha|^{-1} \mu^*(I).
      \end{equation*}
      Of course, the latter two identities still hold if $I = \varnothing$. Hence they hold for all $I \in \sc K$.
  \end{itemize}
  Also, let us use the notation
  \begin{equation*}
    \Lambda(E) = \left\{ \sum_{n=1}^\infty \lambda(I_n); \, I_n \in \sc K, \, \bigcup_{n=1}^\infty I_n \supset E \right\}
  \end{equation*}
  for all $E \subset \bb R$.

  \begin{enumerate}[label=(\alph*)]
    \item Suppose $(I_n)$ is a sequence in $\sc K$ (i.e.\ a sequence of open intervals) and $E \subset \bigcup_n I_n$. Then $T(I_n) \in \sc K$ for every $n$,
      \begin{equation*}
        T(E) \subset T \left( \bigcup_n I_n \right) = \bigcup_n T(I_n),
      \end{equation*}
      and
      \begin{equation*}
        \sum_n \lambda[T(I_n)] = |\alpha| \sum_n \lambda(I_n).
      \end{equation*}
      Thus, if $s \in \Lambda(E)$, then $|\alpha| s \in \Lambda[T(E)]$. It follows that
      \begin{equation*}
        \mu^*[T(E)] = \inf \Lambda[T(E)] \le |\alpha| \inf \Lambda(E) = |\alpha| \mu^*(E).
      \end{equation*}

      Conversely, suppose $(J_n)$ is a sequence in $\sc K$ and $T(E) \subset \bigcup_n J_n$. Then $T^{-1}(J_n) \in \sc K$ for all $n$,
      \begin{equation*}
        E = T^{-1}[T(E)] \subset T^{-1} \left( \bigcup_n J_n \right) = \bigcup_n T^{-1}(J_n),
      \end{equation*}
      and
      \begin{equation*}
        \sum_n \lambda[T^{-1}(J_n)] = |\alpha|^{-1} \sum_n \lambda(J_n).
      \end{equation*}
      Hence, by the same logic as above, we find that $\mu^*(E) \le |\alpha|^{-1} \mu^*[T(E)]$, and it follows that
      \begin{equation*}
        \mu^*[T(E)] = |\alpha| \mu^*(E).
      \end{equation*}

    \item Note that if $f: X \to Y$ is a bijective function (between arbitrary sets $X,Y$), then
      \begin{equation*}
        \begin{split}
          f^{-1}[f(A)] &= A, \\
          f(A \cup B) &= f(A) \cup f(B), \\
          f(A - B) &= f(A) - f(B), \\
          f[f^{-1}(C)] &= C, \\
        \end{split}
      \end{equation*}
      for all $A, B \subset X$ and $C \subset Y$.

      Suppose that $E$ is measureable:
      \begin{equation*}
        \mu^*(A) = \mu^*(A \cap E) + \mu^*(A - E)
      \end{equation*}
      for all $A \subset \bb R$. Then, for all $B \subset \bb R$, we have
      \begin{equation*}
        \begin{split}
          \mu^*[B \cap T(&E)] + \mu^*[B - T(E)] \\
          &= \mu^*[T(T^{-1}(B) \cap E)] + \mu^*[T(T^{-1}(B) - E)] \\
          &= |\alpha| \mu^*[T^{-1}(B) \cap E] + |\alpha| \mu^*[T^{-1}(B) - E] \\
          &= |\alpha| \mu^*[T^{-1}(B)] \\
          &= \mu^*(B),
        \end{split}
      \end{equation*}
      so that $T(E)$ is measurable.

      Conversely, suppose that $T(E)$ is measurable. Then, for all $A \subset \bb R$,
      \begin{equation*}
        \begin{split}
          \mu^*(A \cap E&) + \mu^*(A - E) \\
          &= \mu^*[T^{-1}(T(A) \cap T(E))] + \mu^*[T^{-1}(T(A) - T(E))] \\
          &= |\alpha|^{-1} \mu^*[T(A) \cap T(E)] + |\alpha|^{-1} \mu^*[T(A) - T(E)] \\
          &= |\alpha|^{-1} \mu^*[T(A)] \\
          &= \mu^*(A),
        \end{split}
      \end{equation*}
      so that $E$ is measurable.

    \item This is immediate given (a), (b), and the definition of the Lebesgue-measure. First, $T(E)$ is Lebesgue-measurable by (b). Next, $\mu(E) = \mu^*(E)$ and $\mu[T(E)] = \mu^*[T(E)]$ since $\mu$ is simply the restriction of $\mu^*$ to the measurable sets. Finally, $\mu^*[T(E)] = |\alpha| \mu^*(E)$ by (a).
  \end{enumerate}
\end{solution}

\chapter*{Chapter 2 -- Integration}

\section*{Section 2.1 -- Definition of Measurable Functions}

\subsection*{Problems}

\subsubsection*{2.1.6}
The \emph{characteristic function} of a set $E$ is the function $\chi_E$ defined by
\begin{equation*}
  \chi_E(x) =
  \begin{cases}
    1, & \text{if $x \in E$,} \\
    0, & \text{if $x \notin E$.}
  \end{cases}
\end{equation*}
Prove that the set $E$ is measurable if and only if the function $\chi_E$ is measurable.

\begin{solution}
  Suppose $E \in \cal A$. For all $c \in \bb R$,
  \begin{equation*}
    \chi_E^{-1}\{(-\infty, c)\} = \{x \in X; \chi_E(x) < c\} =
    \begin{cases}
      \varnothing & (c \le 0), \\
      E^c & (0 < c \le 1), \\
      X & (c > 1),
    \end{cases}
  \end{equation*}
  so that $\chi_E^{-1}\{(-\infty, c)\} \in \cal A$. By Theorem 2.1.1, $\chi_E$ is measurable.

  Conversely, suppose $\chi_E$ is measurable. Then $E$ is measurable, since
  \begin{equation*}
    E = X - E^c = \chi^{-1}\{(-\infty, 2)\} - \chi^{-1}\{(-\infty, 1)\}.
  \end{equation*}
\end{solution}

\subsubsection*{2.1.9}
If $f$ is measurable, then $|f|$ and $|f|^2$ are measurable.

\begin{solution}
  If $c \le 0$, then
  \begin{equation*}
    (|f|)^{-1}\{(-\infty, c)\} = (|f|^2)^{-1}\{(-\infty, c)\} = \emptyset \in \cal A,
  \end{equation*}
  since $|f|$ and $|f|^2$ are nonnegative functions.
  
  Let $c > 0$. Then
  \begin{equation*}
    (|f|)^{-1}\{(-\infty, c)\} = \{x \in X; -c < f(x) < c\} = f^{-1}\{(-c,c)\}.
  \end{equation*}
  The set $(-c,c)$ is open, hence $f^{-1}\{(-c,c)\} \in \cal A$ by the measurability of $f$. Similarly,
  \begin{equation*}
    (|f|^2)^{-1}\{(-\infty, c)\} = f^{-1}\{(-\sqrt c, \sqrt c)\} \in \cal A.
  \end{equation*}

  Finally,
  \begin{equation*}
    (|f|)^{-1}\{+\infty\} = (|f|^2)^{-1}\{+\infty\} = f^{-1}\{+\infty\} \cup f^{-1}\{-\infty\} \in \cal A
  \end{equation*}
  by the measurability of $f$, and
  \begin{equation*}
    (|f|)^{-1}\{-\infty\} = (|f|^2)^{-1}\{-\infty\} = \emptyset \in \cal A
  \end{equation*}
  since $|f|$ and $|f|^2$ are nonnegative. Thus, both $|f|$ and $|f|^2$ are measurable by Theorem 2.1.1.
\end{solution}

\subsubsection*{2.1.10}
A monotone function defined on the real line is Lebesgue-measurable.

\begin{solution}
  Let $f$ be a monotone increasing extended real-valued function on $\bb R$;
  \begin{equation*}
    (\forall x,y \in \bb R): \quad x < y \implies f(x) \le f(y).
  \end{equation*}
  Given any $c \in \bb R$, let
  \begin{equation*}
    \xi_c = \inf \{x \in X; f(x) \ge c\}.
  \end{equation*}
  We need to consider two cases: $f(\xi_c) < c$ and $f(\xi_c) \ge c$. In the former case, $f(x) < c$ for all $x \le \xi_c$ and $f(x) \ge c$ for all $x > \xi_c$ (by monotonicity). Hence
  \begin{equation*}
    f^{-1}\{(-\infty,c)\} = (-\infty, \xi_c].
  \end{equation*}
  This is a Borel set, hence also a Lebesgue set (see Problem 1.9.3). In the latter case, $f(x) < c$ for all $x < \xi_c$ and $f(x) \ge c$ for all $x \ge \xi_c$, so that
  \begin{equation*}
    f^{-1}\{(-\infty,c)\} = (-\infty, \xi_c),
  \end{equation*}
  which is Lebesgue-measurable. Since $c$ was arbitrary, we conclude that $f$ is measurable, by Theorem 2.1.1.

  The proof for $f$ monotone decreasing is similar.
\end{solution}

\section*{Section 2.2 -- Operations on Measurable Functions}

\subsection*{Problems}

\subsubsection*{2.2.2}
Let $g(u_1, \dots, u_k)$ be a continuous function in $\bb R^k$, and let $\varphi_1, \dots, \varphi_k$ be measurable functions. Prove that the composite function $h(x) = g[\varphi_1(x), \dots, \varphi_k(x)]$ is a measurable function. Note that as a special case we may conclude that
\begin{equation*}
  \max(\varphi, \dots, \varphi_n) \quad \text{and} \quad \min(\varphi, \dots, \varphi_n)
\end{equation*}
are measurable functions.

\begin{solution}
  We will use the following fact, which may be proven in a course in topology:
  \begin{quote}
    $\bb R^k$ has a countable basis of product open subsets. Hence, if $U$ is an open subset of $\bb R^k$, then there are open subsets $U_{ni} \subset \bb R$ for $n = 1, 2, \dots$ and $i = 1, \dots, k$ such that
    \begin{equation*}
      U = \bigcup_{n=1}^\infty (U_{n1} \times \dots \times U_{nk}).
    \end{equation*}
  \end{quote}

  We are assuming that $g$ is real-valued, likewise for the functions $\varphi_i$. Let $c \in \bb R$. Note that $g^{-1}\{(-\infty,c)\}$ is open by continuity of $g$. Thus
  \begin{equation*}
    g^{-1}\{(-\infty,c)\} = \bigcup_{n=1}^\infty (U_{n1} \times \dots \times U_{nk})
  \end{equation*}
  for some open subsets $U_{ni} \subset \bb R$. Hence
  \begin{equation*}
    \begin{split}
      h^{-1}\{(-\infty,c)\} &= \{x \in X; g(\varphi_1(x), \dots, \varphi_k(x)) \le c\} \\
      &= \{x \in X; (\varphi_1(x), \dots, \varphi_k(x)) \in g^{-1}\{(-\infty,c)\}\} \\
      &= \bigcup_{n=1}^\infty \{x \in X; (\varphi_1(x), \dots, \varphi_k(x)) \in U_{n1} \times \dots \times U_{nk}\} \\
      &= \bigcup_{n=1}^\infty \bigcap_{i=1}^k \{x \in X; \varphi_i(x) \in U_{ni}\} \\
      &= \bigcup_{n=1}^\infty \bigcap_{i=1}^k \varphi_i^{-1}(U_{ni}).
    \end{split}
  \end{equation*}
  The sets $\varphi_i(U_{ni})$ are measurable since the functions $\varphi_i$ are measurable. It follows that $h^{-1}\{(-\infty,c)\}$ is measurable, and thus that $h$ is measurable, by Theorem 2.1.1.

  To apply the above to the $\max$ and $\min$ functions $\bb R^k \to \bb R$ we must show that they are continuous. Let $a < b$ and note that
  \begin{equation*}
    \begin{split}
      {\max}^{-1}\{(a,b)\} &= \{(x_1, \dots, x_k) \in \bb R^k; \text{$x_i > a$ for some $i$}\} \\
      &\quad \cap \{(x_1, \dots, x_k) \in \bb R^k; \text{$x_i < b$ for all $i$}\}.
    \end{split}
  \end{equation*}
  Both sets in the above binary intersection are easily seen to be open by considering $\epsilon$-neighborhoods about their points. It follows that $\max^{-1}(U)$ is open for all open subsets $U \in \bb R^k$, since every such $U$ can be written as a countable union of open intervals. Thus $\max$ is continuous, and one similarly shows that $\min$ is continuous.
\end{solution}

\subsubsection*{2.2.3}
Let $f(x)$ be a measurable function and define
\begin{equation*}
  g(x) =
  \begin{cases}
    \frac{1}{f(x)}, & \text{if $f(x) \ne 0$,} \\
    0, & \text{if $f(x) = 0$.}
  \end{cases}
\end{equation*}
Prove that $g$ is measurable.

\begin{solution}
  For $c < 0$,
  \begin{equation*}
    g^{-1}\{(-\infty,c)\} = \{x; 1/f(x) < c\} = \{x; 1/c < f < 0\} = f^{-1}\{(1/c, 0)\},
  \end{equation*}
  which is measurable by the measurability of $f$. Next,
  \begin{equation*}
    g^{-1}\{(-\infty,0)\} = \{x; 1/f(x) < 0\} = \{x; f(x) < 0\} = f^{-1}\{(-\infty, 0)\},
  \end{equation*}
  also measurable. Note that if we take the natural convention (unfortunately not addressed in the text) that $x/(\pm \infty) = 0$ for all $x \in \bb R$, then
  \begin{equation*}
    g^{-1}(\{0\}) = \{x; f(x) = 0\} \cup \{x; f(x) = \pm \infty\} = f^{-1}(\{0\}) \cup f^{-1}(\{\pm \infty\}).
  \end{equation*}
  Hence, for $c > 0$,
  \begin{equation*}
    \begin{split}
      g^{-1}\{(-\infty,c)\} &= g^{-1}\{(-\infty,0)\} \cup g^{-1}(\{0\}) \cup g^{-1}\{(0,c)\} \\
      &= f^{-1}\{(-\infty, 0)\} \cup f^{-1}(\{0\}) \cup f^{-1}(\{\pm \infty\}) \cup f^{-1}\{(1/c, \infty)\} \\
      &= f^{-1}\{(-\infty, 0]\} \cup f^{-1}(\{\pm \infty\}) \cup f^{-1}\{(1/c, \infty)\},
    \end{split}
  \end{equation*}
  which is measurable by the measurability of $f$ (see Problem 2.1.4). Finally, $g^{-1}(\{\pm \infty\}) = \emptyset$, and it follows by Theorem 2.1.1 that $g$ is measurable.
\end{solution}

\section*{Section 2.3 -- Egoroff's Theorem}

\subsection*{Problems}

\subsubsection*{2.3.2}
Let $\{f_n\}$ be a sequence of measurable functions in a finite measure space $X$. Suppose that for almost every $x$, $\{f_n(x)\}$ is a bounded set. Then for any $\epsilon > 0$ there exist a positive number $c$ and a measurable set $E$ with $\mu(X - E) < \epsilon$, such that $|f_n(x)| \le c$ for all $x \in E$, $n = 1,2, \dots$.

\begin{solution}
  The definition we have for `bounded set' applies to metric spaces, and it does not make much sense here since the functions $f_n$ may be extended real-valued. Hence we will assume that `$\{f_n(x)\}$ is a bounded set' means that $\sup_n |f_n(x)| < \infty$.

  Let $g = \sup_n |f_n|$, and note that $g$ is measurable by Problem 2.1.9 and Theorem 2.2.3. Let $F = \{x; \, g(x) < \infty\}$. Notice that $g(x) < \infty$ if and only if $\{f_n(x)\}$ is bounded. Hence $\mu(X - F) = 0$.

  For $k = 1,2,\dots$, define $F_k = \{x; \, g(x) \le k\}$. Then $F_1 \subset F_2 \subset \dotsm$ and $\lim_k F_k = \bigcup_{k=1}^\infty F_k = F$.  By Theorem 1.2.1(iv),
  \begin{equation*}
    \lim_k \mu(X - F_k) = \mu(X - F) = 0.
  \end{equation*}
  Given any $\epsilon > 0$, there exists a positive integer $K$ such that $\mu(X - F_k) < \epsilon$ for all $k \ge K$. In particular $\mu(X - F_K) < \epsilon$, and $g(x) \le K$ for all $x \in F_K$, which means that $|f_n(x)| \le K$ for all $x \in F_K$.
\end{solution}

\section*{Section 2.4 -- Convergence in Measure}

\subsection*{Problems}

\subsubsection*{2.4.3}
Prove the following result (which immediately yields another proof of Corollary 2.4.2): Let $f_n$ ($n = 1,2,\dots$) and $f$ be a.e.\ real-valued measurable functions in a finite measure space. For any $\epsilon > 0$, $n \ge 1$, let
\begin{equation*}
  E_n(\epsilon) = \{x; \, |f_n(x) - f(x)| \ge \epsilon\}.
\end{equation*}
Then $\{f_n\}$ converges a.e.\ to $f$ if and only if
\begin{equation*}
  \lim_{n \to \infty} \mu \left[ \bigcup_{m=n}^\infty E_m(\epsilon) \right] = 0 \quad \text{for any $\epsilon > 0$.} \tag{2.4.2}
\end{equation*}
[\emph{Hint:} Let $F = \{x; \, \text{$\{f_n(x)\}$ is not convergent to $f(x)$}\}$. Then \\ $F = \bigcup_{k=1}^\infty \varlimsup_n E_n(1/k)$. Show that $\mu(F) = 0$ if and only if (2.4.2) holds.]

\begin{solution}
  Define
  \begin{equation*}
    F = \bigcup_{k=1}^\infty \varlimsup_n E_n \left( \frac{1}{k} \right) = \bigcup_{k=1}^\infty \bigcap_{n=1}^\infty \bigcup_{m=n}^\infty E_m \left( \frac{1}{k} \right).
  \end{equation*}
  Note that
  \begin{equation*}
    \begin{split}
      x \in F &\iff \exists k, \forall n, \exists m \ge n, \, |f_m(x) - f(x)| \ge \frac{1}{k} \\
      &\iff \neg \left( \forall k, \exists n, \forall m \ge n, \, |f_m(x) - f(x)| < \frac{1}{k} \right) \\
      &\iff f_n(x) \not\to f(x),
    \end{split}
  \end{equation*}
  so that
  \begin{equation*}
    F = \{x; \, f_n(x) \not\to f(x)\}.
  \end{equation*}

  Suppose (2.4.2) holds. Fix $\delta > 0$. For every positive integer $k$, there exists a positive integer $n_k$ such that $n \ge n_k$ implies
  \begin{equation*}
    \mu \left[ \bigcup_{m=n}^\infty E_m \left( \frac{1}{k} \right) \right] < \frac{\delta}{2^k}.
  \end{equation*}
  By subadditivity and monotonicity,
  \begin{equation*}
    \begin{split}
      \mu(F) &= \mu \left[ \bigcup_{k=1}^\infty \bigcap_{n=1}^\infty \bigcup_{m=n}^\infty E_m \left( \frac{1}{k} \right) \right] \le \sum_{k=1}^\infty \mu \left[ \bigcap_{n=1}^\infty \bigcup_{m=n}^\infty E_m \left( \frac{1}{k} \right) \right] \\
      &\le \sum_{k=1}^\infty \mu \left[ \bigcup_{m=n_k}^\infty E_m \left( \frac{1}{k} \right) \right] < \sum_{k=1}^\infty \frac{\delta}{2^k} = \delta.
    \end{split}
  \end{equation*}
  Since $\delta$ was arbitrary, $\mu(F) = 0$, and it follows that $f_n \to f$ a.e.

  Conversely, suppose $f_n \to f$ a.e., so that $\mu(F) = 0$. By monotonicity and Theorem 1.2.2,
  \begin{equation*}
    0 = \mu(F) = \mu \left[ \bigcup_{k=1}^\infty \varlimsup_n E_n \left( \frac{1}{k} \right) \right] \ge \mu \left[ \varlimsup_n E_n \left( \frac{1}{l} \right) \right] \ge \varlimsup_n \mu \left[ E_n \left( \frac{1}{l} \right) \right]
  \end{equation*}
  for all positive integers $l$. But of course $\varlimsup_n \mu \left[ E_n \left( 1/l \right) \right] \ge \varliminf_n \mu \left[ E_n \left( 1/l \right) \right] \ge 0$ since $\mu$ is nonnegative, so $\lim_n \mu \left[ E_n \left( 1/l \right) \right]$ exists and is equal to zero. Note that the sets $\bigcup_{m=n}^\infty E_m(1/l)$ are decreasing, so their limit as $n \to \infty$ exists. Hence we can apply Corollary 1.2.3 and monotonicity to find that
  \begin{equation*}
    \begin{split}
      \lim_n \mu \left[ \bigcup_{m=n}^\infty E_m \left( \frac{1}{l} \right) \right] &= \mu \left[ \lim_n \bigcup_{m=n}^\infty E_m \left( \frac{1}{l} \right) \right] = \mu \left[ \bigcap_{n=1}^\infty \bigcup_{m=n}^\infty E_m \left( \frac{1}{l} \right) \right] \\
      &\le \mu \left[\bigcup_{k=1}^\infty \bigcap_{n=1}^\infty \bigcup_{m=n}^\infty E_m \left( \frac{1}{k} \right) \right] = \mu(F) = 0.
    \end{split}
  \end{equation*}
  Finally, given $\epsilon > 0$, note that
  \begin{equation*}
    E_n(\epsilon) \subset E_n \left( \frac{1}{\lceil 1/\epsilon \rceil} \right).
  \end{equation*}
  Hence
  \begin{equation*}
    \lim_n \mu \left[ \bigcup_{m=n}^\infty E_m(\epsilon) \right] \le \lim_n \mu \left[ \bigcup_{m=n}^\infty E_m \left( \frac{1}{\lceil 1/\epsilon \rceil} \right) \right] \le 0
  \end{equation*}
  by monotonicity, and (2.4.2) follows.
\end{solution}

\subsubsection*{2.4.4}
Let $X$ be the set of all positive integers, $\cal A$ the class of all subsets of $X$, and $\mu(E)$ (for any $E \in \cal A$) the number of points in $E$. Prove that in this measure space, convergence in measure is equivalent to uniform convergence.

\begin{solution}
  Uniform convergence always implies convergence in measure. Conversely, suppose $(f_n)$ converges in measure to $f$. Given any $\epsilon > 0$, there exists a positive integer $N$ such that $n \ge N$ implies
  \begin{equation*}
    \mu \left[ \{x; \, |f_n(x) - f(x)| \ge \epsilon\} \right] < 1.
  \end{equation*}
  That is, for $n \ge N$ the set $\{x; \, |f_n(x) - f(x)| \ge \epsilon\}$ is empty, which in particular means that $\sup_x |f_n(x) - f(x)| \le \epsilon$. It follows that $f_n \to f$ uniformly.
\end{solution}

\section*{Section 2.5 -- Integrals of Simple Functions}

\subsection*{Problems}

\subsubsection*{2.5.2}
An integrable simple function $f$ is equal a.e.\ to zero if and only if $\int_E f d\mu = 0$ for any measurable set $E$.

\begin{solution}
  Let $f$ be an integrable simple function. Then $f$ can be written in the form
  \begin{equation*}
    f = \sum_{i=1}^n \alpha_i \chi_{E_i},
  \end{equation*}
  for mutually disjoint sets $E_1, \dots E_n$, with all $\alpha_i \ne 0$, and all $\mu(E_i) < \infty$.

  Suppose $f = 0$ a.e., and let $E$ be any measurable set. By Theorem 2.5.1(b) and (g),
  \begin{equation*}
    0 \le \int_E f d\mu \le \int f d\mu = \sum_{i=1}^n \alpha_i \mu(E_i).
  \end{equation*}
  But $\mu(E_i) = 0$ since $f = 0$ a.e., so $\int_E f d\mu = 0$.

  Conversely, suppose $\int_E f d\mu = 0$ for all measurable sets $E$. Then
  \begin{equation*}
    \alpha_i \mu(E_i) = \int_{E_i} f d\mu = 0,
  \end{equation*}
  so that $\mu(E_i) = 0$, for all $i \in \{1,\dots,n\}$. It follows that $f = 0$ a.e.
\end{solution}

\section*{Section 2.6 -- Definition of the Integral}

\subsection*{Problems}

\subsubsection*{2.6.3}
Let $f$ be a measurable function. Prove that $f$ is integrable if and only if $f^+$ and $f^-$ are integrable, or if and only if $|f|$ is integrable.

\begin{solution}
  Let $f$ be measurable. We must prove the equivalence of the following statements:
  \begin{enumerate}[label=(\roman*)]
    \item $f$ is integrable.
    \item $f^+$ and $f^-$ are integrable.
    \item $|f|$ is integrable.
  \end{enumerate}
  We will first show that (iii)$\implies$(ii), then that (ii)$\implies$(i), and finally that (i)$\implies$(iii).

  Suppose that $|f|$ is integrable. Let $E = \{x; \, f(x) \ge 0\} = f^{-1}{[0,\infty)}$, and note that $E$ is measurable since $f$ is. There exists a Cauchy in the mean sequence $(g_n)$ of integrable simple functions converging to $|f|$ a.e., and the sequence $(\chi_E g_n)$ is easily seen to satisfy the corresponding properties with respect to $f^+$. Since $f^+$ is measurable by Problem 2.1.8, this implies that it is integrable. The proof that $f^-$ is integrable is similar.

  Next, suppose that $f^+$ and $f^-$ are integrable. Then there exist Cauchy in the mean sequences $(g_n)$ and $(h_n)$ of integrable simple functions converging a.e.\ to $f^+$ and $f^-$, respectively. Define a new sequence $(f_n)$ of integrable simple functions by $f_n = g_n - h_n$. Then $(f_n)$ is Cachy in the mean, since
  \begin{equation*}
    |f_n - f_m| = |g_n - h_n - g_m + h_m| \le |g_n - g_m| + |h_n - h_m|.
  \end{equation*}
  It also converges to $f$ a.e.\, since
  \begin{equation*}
    |f_n - f| = |g_n - h_n - f^+ + f^-| \le |g_n - f^+| + |h_n - f^-|.
  \end{equation*}
  It follows that $f$ is integrable.

  Finally, assume that $f$ is integrable. There is a Cauchy in the mean sequence $(f_n)$ of integrable simple functions converging to $f$ a.e. The sequence $(|f_n|)$ consists of integrable simple functions. It is Cauchy in the mean since
  \begin{equation*}
    ||f_n| - |f_m|| \le |f_n - f_m|,
  \end{equation*}
  and it converges to $|f|$ a.e.\ since
  \begin{equation*}
    ||f_n| - |f|| \le |f_n - f|.
  \end{equation*}
  Since $|f|$ is measurable by Problem 2.1.9, it follows that $|f|$ is integrable.
\end{solution}

\subsubsection*{2.6.4}
Let $X$ be the measure space described in Problem 2.4.4. Then $f$ is integrable if and only if the series $\sum_{n=1}^\infty |f(n)|$ is convergent. If $f$ is integrable, then
\begin{equation*}
  \int f \, d\mu = \sum_{n=1}^\infty f(n).
\end{equation*}

\begin{solution}
  Suppose $f$ is integrable. Then there is a Cauchy in the mean sequence $(f_n)$ of integrable simple functions converging to $f$ a.e. We saw in the previous problem that this implies that $|f|$ is integrable, and that $(|f_n|)$ is a Cauchy in the mean sequence of integrable simple functions converging to $|f|$ a.e. Note that in this particular space convergence a.e.\ is the same as convergence everywhere (since the only subset with measure zero is $\emptyset$).

  By Theorem 2.5.1(h),
  \begin{equation*}
    \int |f_n| \, d\mu = \sum_{i=1}^\infty \int_{\{i\}} |f_n| \, d\mu = \sum_{i=1}^\infty |f_n(i)|.
  \end{equation*}
  Hence, in particular,
  \begin{equation*}
    \int |f| \, d\mu = \lim_{n \to \infty} \sum_{i=1}^\infty |f_n(i)|.
  \end{equation*}

  Given any positive integer $m$, there exists $n'$ such that
  \begin{equation*}
    {|f(i) - f_{n'}(i)|} < 1/m \quad (i = 1, 2, \dots, m)
  \end{equation*}
  (since $f_n \to f$) and
  \begin{equation*}
    \left| \sum_{i=1}^\infty |f_{n'}(i)| - \int |f| \, d\mu \right| < 1
  \end{equation*}
  (since $\sum_i |f_n(i)| \to \int |f| \, d\mu$). Thus
  \begin{equation*}
    \sum_{i=1}^m |f(i)| \le \sum_{i=1}^m |f(i) - f_{n'}(i)| + \sum_{i=1}^m |f_{n'}(i)| < 1 + \sum_{i=1}^\infty |f_{n'}(i)| < 2 + \int |f| \, d\mu,
  \end{equation*}
  and it follows that the series $\sum_{i=1}^\infty |f(i)|$ converges (to a finite number).

  Conversely, assume that the series $\sum_{i=1}^\infty |f(i)|$ converges. Define a sequence of integrable simple functions $(g_n)$ by
  \begin{equation*}
    g_n = \sum_{i=1}^n f(i) \chi_{\{i\}}.
  \end{equation*}
  It is clear that $g_n \to f$ everywhere. Moreover, if $m > n$, then
  \begin{equation*}
    \int |g_m - g_n| \, d\mu = \int \left| \sum_{i=n+1}^m f(i) \chi_{\{i\}} \right| d\mu = \sum_{n+1}^m |f(i)| \le \sum_{n+1}^\infty |f(i)|.
  \end{equation*}
  The right-hand side goes to zero as $n \to \infty$ since $\sum_{i=1}^\infty |f(i)|$ is convergent, which means that $\int |g_m - g_n| \, d\mu \to 0$ as $n,m \to \infty$; i.e., $(g_n)$ is Cauchy in the mean. It follows that $f$ is integrable, with
  \begin{equation*}
    \int f \, d\mu = \lim_{n \to \infty} \int g_n \, d\mu = \lim_{n \to \infty} \sum_{i=1}^n f(i) = \sum_{i=1}^\infty f(i).
  \end{equation*}
\end{solution}


\chapter*{Chapter 3 -- Metric Spaces}

\section*{Section 3.1 -- Topological and Metric Spaces}

\subsection*{Problems}

\subsubsection*{3.1.1}
Prove that if $(X,p)$ is a metric space, and if
\begin{equation*}
  \hat \rho(x,y) = \frac{\rho(x,y)}{1 + \rho(x,y)},
\end{equation*}
then also $(X,\hat \rho)$ is a metric space. [\emph{Hint:} Cf.\ the proof of (3.1.3).]

\begin{solution}
  The only nonobvious property is the triangle inequality. Let $x,y,z$ be arbitrary points of $X$. Since $t \mapsto t/(1+t)$ is monotone increasing on $[0,\infty)$, and since $\rho(x,z) \le \rho(x,y) + \rho(y,z)$, we have
  \begin{equation*}
    \frac{\rho(x,z)}{1 + \rho(x,z)} \le \frac{\rho(x,y) + \rho(y,z)}{1 + \rho(x,y) + \rho(y,z)}.
  \end{equation*}
  Moreover, by equation (3.1.3),
  \begin{equation*}
    \frac{\rho(x,y) + \rho(y,z)}{1 + \rho(x,y) + \rho(y,z)} \le \frac{\rho(x,y)}{1 + \rho(x,y)} + \frac{\rho(y,z)}{1 + \rho(y,z)}.
  \end{equation*}
  It follows that $\hat \rho(x,z) \le \hat \rho(x,y) + \rho(y,z)$.
\end{solution}

\subsubsection*{3.1.2}
Let $X, \rho, \hat \rho$ be as in Problem 3.1.1. Prove that $\rho(x_n, x) \to 0$ if and only if $\hat \rho(x_n, x) \to 0$. Give an example showing that $\rho$ and $\hat \rho$ are not equivalent in general.

\begin{solution}
  It is clear that $\rho(x_n, x) \to 0$ implies $\hat \rho (x_n, x) \to 0$, since $\hat \rho \le \rho$.
  
  Conversely, suppose $\hat \rho(x_n, x) \to 0$. Given any $\epsilon > 0$, there is a positive integer $N$ such that
  \begin{equation*}
    \hat \rho(x_n, x) < \frac{\epsilon}{1 + \epsilon} \quad (n \ge N).
  \end{equation*}
  Substituting the definition of $\hat \rho$ and rearranging yields $\rho(x_n, x) < \epsilon$.

  If $\rho$ and $\hat \rho$ are equivalent then, in particular, there exists a positive constant $\beta$ such that
  \begin{equation*}
    \frac{\rho(x,y)}{\hat \rho(x,y)} \le \beta
  \end{equation*}
  whenever $x \ne y$. But
  \begin{equation*}
    \frac{\rho(x,y)}{\hat \rho(x,y)} = 1 + \rho(x,y),
  \end{equation*}
  so this is impossible if $X$ is unbounded (w.r.t.\ $\rho$), say if $X = \bb R^n$ and $\rho$ is the Euclidean metric.
\end{solution}

\subsubsection*{3.1.6}
Prove that the spaces $l^1, s, c, c_0$ are separable metric spaces.

\begin{solution}
  For each of the spaces $X$ we will take an arbitrary element $x \in X$ and demonstrate that every $\epsilon$-ball around $x$ contains a point $y$ of a certain countable subset $Y \subset X$. It will then follow that $Y$ is dense in $X$, and hence that $X$ is separable.

  Let $x = (x_i) \in l^1$. Fix $\epsilon > 0$. Since $\sum_i |x_i| < \infty$, there exists $n$ such that
  \begin{equation*}
    \sum_{i=n+1}^\infty |x_i| < \frac{\epsilon}{2}.
  \end{equation*}
  For $i = 1, \dots, n$, choose $y_i \in \bb Q$ (or $y_i$ with rational real and imaginary parts in the complex case) such that $|x_i - y_i| < \epsilon/2n$, and let $y = (y_1, \dots, y_n, 0, 0, \dots)$. Then
  \begin{equation*}
    \rho(x,y) = \sum_{i=1}^n |x_i - y_i| + \sum_{i=n+1}^\infty |x_i| < \epsilon.
  \end{equation*}
  Moreover, $y$ is an element of the subset of $l^1$ consisting of sequences with rational components, with only finitely many being nonzero. This subset is easily seen to be countable, and it follows that $l^1$ is separable.

  Next, let $x = (x_i) \in s$, and fix $\epsilon > 0$. Choose $n$ such that
  \begin{equation*}
    \sum_{i=n+1}^\infty \frac{1}{2^i} \frac{|x_i|}{1 + |x_i|} < \frac{\epsilon}{2}.
  \end{equation*}
  For $i = 1, \dots, n$, choose $y_i \in \bb Q$ such that $|x_i - y_i| < \epsilon/2$, and let $y = (y_1, \dots, y_n, 0, 0, \dots)$. Then
  \begin{equation*}
    \rho(x,y) = \sum_{i=1}^n \frac{1}{2^i} \frac{|x_i - y_i|}{1 + |x_i - y_i|} + \sum_{i=n+1}^\infty \frac{1}{2^i} \frac{|x_i|}{1 + |x_i|} < \epsilon.
  \end{equation*}
  Similarly to the above, it follows that $s$ is dense.

  Finally, let $x = (x_i) \in c$. (This argument will also cover $c_0$.) Let $\xi = \lim_i x_i$ and fix $\epsilon > 0$. Choose $n$ such that $|x_i - \xi| < \epsilon/2$ for all $i \ge n$. For $i = 1, \dots, n-1$, choose $y_i \in \bb Q$ such that $|x_i - y_i| < \epsilon$. Also choose $\eta \in \bb Q$ such that $|\xi - \eta| < \epsilon/2$ (take $\eta = \xi = 0$ in the $c_0$ case), so that
  \begin{equation*}
    |x_i - \eta| \le |x_i - \xi| + |\xi - \eta| < \epsilon
  \end{equation*}
  for all $i \ge n$. Let $y = (y_1, \dots, y_{n-1}, \eta, \eta, \dots)$. Then
  \begin{equation*}
    \rho(x,y) = \sup_i |x_i - y_i| \le \epsilon.
  \end{equation*}
  It follows that $c$ is separable (and $c_0$ as well).
\end{solution}

\end{document}
