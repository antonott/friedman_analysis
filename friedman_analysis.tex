\documentclass{report}
\usepackage[utf8]{inputenc}
\usepackage{amsmath}
\usepackage{amssymb}
\usepackage{amsthm}
\usepackage{mathrsfs}
\usepackage{enumitem}
\usepackage{array}

\newcommand{\bb}[1]{\mathbb{#1}}
\newcommand{\norm}[1]{\lVert #1 \rVert}
\newcommand{\brac}[1]{\langle #1 \rangle}

\let\sc\relax
\newcommand{\sc}[1]{\mathscr{#1}}

\let\cal\relax
\newcommand{\cal}[1]{\mathcal{#1}}

\theoremstyle{remark}
\newtheorem*{solution}{Solution}

\setenumerate{listparindent=\parindent, parsep=0pt}

\title{Notes for \textit{Foundations of Modern Analysis} by Avner Friedman}
\author{Anton Ottosson -- antonott@kth.se}
\date{\today}

\begin{document}

\maketitle

\chapter*{Chapter 1 -- Measure theory}

\section*{Section 1.1 -- Rings and algebras}

\subsection*{Problems}

\subsubsection*{1.1.1}
\begin{equation*}
  \left( \varliminf_n E_n \right)^c = \varlimsup_n E_n^c, \quad \left( \varlimsup_n E_n \right)^c = \varliminf_n E_n^c.
\end{equation*}

\begin{solution}
  Note that
  \begin{equation*}
    \begin{split}
      x \in \varliminf_n E_n &\iff \text{$x \in E_n$ for all but finitely many $n$} \\
      &\iff \text{$x \in E_n^c$ for only finitely many $n$}.
    \end{split}
  \end{equation*}
  Hence
  \begin{equation*}
    \begin{split}
      x \in \left( \varliminf_n E_n \right)^c &\iff \text{$x \in E_n^c$ for infinitely many $n$} \\
      &\iff x \in \varlimsup_n E_n^c,
    \end{split}
  \end{equation*}
  proving the first identity.

  Next, let $F_n = E_n^c$ for every $n$. Then
  \begin{equation*}
    \varlimsup_n E_n = \varlimsup_n F_n^c = \left( \varliminf_n F_n \right)^c = \left( \varliminf_n E_n^c \right)^c
  \end{equation*}
  by the first identity, and the second identity follows.
\end{solution}

\subsubsection{1.1.2}
\begin{equation*}
  \varlimsup_n E_n = \bigcap_{k=1}^\infty \bigcup_{n=k}^\infty E_n, \quad \varliminf_n E_n = \bigcup_{k=1}^\infty \bigcap_{n=k}^\infty E_n.
\end{equation*}

\begin{solution}
  Suppose $x \in \varlimsup_n E_n$. Then $x \in E_n$ for infinitely many $n$. It follows that $x \in \bigcup_{n=k}^\infty E_n$ for all $k \in \bb N$, and hence that $x \in \bigcap_{k=1}^\infty \bigcup_{n=k}^\infty E_n$.

  Conversely, assume that $x \in \bigcap_{k=1}^\infty \bigcup_{n=k}^\infty E_n$. Then $x \in \bigcup_{n=k}^\infty E_n$ for all $k \in \bb N$. It follows that $x \in E_n$ for infinitely many $n$, and thus that $x \in \varlimsup_n E_n$. This proves the first identity.

  Next, suppose that $x \in \varliminf_n E_n$. Then $x \in E_n$ for all but finitely many $n$, so there is some $k' \in \bb N$ such that $x \in E_n$ for all $n \ge k'$. It follows that $x \in \bigcap_{n=k'}^\infty E_n$, and hence that $x \in \bigcup_{k=1}^\infty \bigcap_{n=k}^\infty E_n$.

  Conversely, assume that $x \in \bigcup_{k=1}^\infty \bigcap_{n=k}^\infty E_n$. Then $x \in \bigcap_{n=k'}^\infty E_n$ for some $k' \in \bb N$, which means that $x \in E_n$ for all $n \ge k'$. It follows that $x \in E_n$ for all but finitely many $n$; that is, $x \in \varliminf_n E_n$.
\end{solution}

\subsubsection*{1.1.3}
If $\sc R$ is a $\sigma$-ring and $E_n \in \sc R$, then
\begin{equation*}
  \bigcap_{n=1}^\infty E_n \in \sc R, \quad \varlimsup_n E_n \in \sc R, \quad \varliminf_n E_n \in \sc R.
\end{equation*}

\begin{solution}
  Let $Y = \bigcup_{n=1}^\infty E_n \in \sc R$. Then $E_n \subset Y$ for all $Y$, and it follows that
  \begin{equation*}
    \bigcap_{n=1}^\infty E_n = Y \cap \left( \bigcap_{n=1}^\infty E_n \right) = Y - \left( Y - \bigcap_{n=1}^\infty E_n \right).
  \end{equation*}
  Notice that
  \begin{equation*}
    Y - \bigcap_{n=1}^\infty E_n = \bigcup_{n=1}^\infty (Y - E_n) \in \sc R,
  \end{equation*}
  by properties (b) and (e). (The equality is analogous to the identity (1.1.2), but with $Y$ in place of $X$.) It follows (by (b) again) that $\bigcap_{n=1}^\infty E_n \in \sc R$. For later reference, let us call this result (x).

  Given $k \in \bb N$, let $A_n = \varnothing$ for $n < k$, and let $A_n = E_n$ for $n \ge k$. Then $A_n \in \sc R$ for all $n$ by (a), hence
  \begin{equation*}
    \bigcup_{n=k}^\infty E_n = \bigcup_{n=1}^\infty A_n \in \sc R
  \end{equation*}
  by (e). It then follows by (x) that
  \begin{equation*}
    \varlimsup_n E_n = \bigcap_{k=1}^\infty \bigcup_{n=k}^\infty E_n \in \sc R.
  \end{equation*}

  By a similar argument we find that (x) implies
  \begin{equation*}
    \bigcap_{n=k}^\infty E_n \in \sc R
  \end{equation*}
  for all $k \in \bb N$. Thus
  \begin{equation*}
    \varliminf_n E_n = \bigcup_{k=1}^\infty \bigcap_{n=k}^\infty E_n \in \sc R
  \end{equation*}
  by (e).
\end{solution}

\subsubsection*{1.1.4}
The intersection of any collection of rings (algebras, $\sigma$-rings, or $\sigma$-algebras) is also a ring (an algebra, $\sigma$-ring, or $\sigma$-algebra).

\begin{solution}
  Let $\sc C$ be a collection of classes. Let $\bigcap \sc C$ denote the intersection of all classes in $\sc C$. We will show that if one of the properties (a)-(e) is satisfied by all classes in $\sc C$, then $\bigcap \sc C$ satisfies that property as well. The result requested in the problem then follows as an immediate corollary.

  It is clear that if every $\sc R \in \sc C$ satisfies (a), then so does $\bigcap \sc C$. Suppose every $\sc R \in \sc C$ satisfies (b). If $A,B \in \bigcap \sc C$ then $A,B \in \sc R$ for every $\sc R \in \sc C$. Hence $A - B \in \sc R$ for all $\sc R \in \sc C$, and it follows that $A - B \in \bigcap \sc C$. The argument for (c) is similar (with $A \cup B$ in place of $A - B$), and (d) is obvious.

  Finally, suppose that every $\sc R \in \sc C$ satisfies (e). If $A_1, A_2, \dotsc \in \bigcap \sc C$ then $A_1, A_2, \dotsc \in \sc R$ for every $\sc R \in \sc C$. Hence $\bigcup_{n=1}^\infty A_n \in \sc R$ for all $\sc R \in \sc C$, and it follows that $\bigcup_{n=1}^\infty A_n \in \bigcap \sc C$.
\end{solution}

\subsubsection*{1.1.5}
If $\sc D$ is any class of sets, then there exists a unique ring $\sc R_0$ such that (i) $\sc R_0 \supset \sc D$, and (ii) any ring $\sc R$ containing $\sc D$ contains also $\sc R_0$. $\sc R_0$ is called the \emph{ring generated} by $\sc D$, and is denoted by $\sc R(\sc D)$.

\begin{solution}
  Let $\sc R_0$ be the intersection of all rings containing $\sc D$. This is a ring by the previous exercise, and it satisfies the properties (i) and (ii). To see that it is unique, let $\sc R_0'$ also by a ring satisfying (i) and (ii). Then $\sc R_0 \subset \sc R_0'$ and $\sc R_0' \subset \sc R_0$ by property (ii), so $\sc R_0 = \sc R_0'$.
\end{solution}

\subsubsection*{1.1.6}
If $\sc D$ is any class of sets, then there exists a unique $\sigma$-ring $\sc S_0$ such that (i) $\sc S_0 \supset \sc D$, and (ii) any $\sigma$-ring containing $\sc D$ contains also $\sc S_0$. We call $\sc S_0$ the \emph{$\sigma$-ring generated} by $\sc D$, and denote it by $\sc S(\sc D)$. A similar result holds for $\sigma$-algebras, and we speak of the \emph{$\sigma$-algebra generated} by $\sc D$.

\begin{solution}
  By the same argument as in the previous exercise, $\sc S_0$ is the intersection of all $\sigma$-rings containing $\sc D$. Similarly the $\sigma$-algebra generated by $\sc D$ is the intersection of all $\sigma$-algebras containing $\sc D$.
\end{solution}

\subsubsection*{1.1.7}
If $\sc D$ is any class of sets, then every set in $\sc R(\sc D)$ can be covered by (that is, is contained in) a finite union of sets of $\sc D$. [\emph{Hint:} The class $\sc K$ of sets that can be covered by finite unions of sets of $\sc D$ forms a ring.]

\begin{solution}
  Let $\sc K$ be the class of all sets that can be covered by a finite union of sets in $\sc D$. Certainly $\varnothing \in \sc K$, since $\varnothing$ is a subset of the empty union. If $A, B \in \sc K$, then
  \begin{equation*}
    A \subset \bigcup_{i=1}^m E_i, \quad B \subset \bigcup_{i=1}^n F_i,
  \end{equation*}
  for some sets $E_1, \dots, E_m, F_1, \dots, F_n \in \sc D$. (Note that $m$ or $n$ can be zero, in which case the corresponding union is empty.) Thus
  \begin{equation*}
    A - B \subset A \subset \bigcup_{i=1}^m E_i
  \end{equation*}
  and
  \begin{equation*}
    A \cup B \subset \left( \bigcup_{i=1}^m E_i \right) \cup \left( \bigcup_{j=1}^n F_j \right),
  \end{equation*}
  so both $A - B$ and $A \cup B$ are elements of $\sc K$.

  The above shows that $\sc K$ is a ring, and certainly $\sc D \subset \sc K$. Hence $\sc R(\sc D) \subset \sc K$ by Problem 1.1.5, and it follows that every set in $\sc R(\sc D)$ can be covered by a finite union of sets in $\sc D$.
\end{solution}

\section*{Section 1.1 -- Definition of measure}

\subsection*{Problems}

\subsubsection*{1.2.1}
If $\mu$ satisfies the properties (i)-(iii) in Definition 1.2.1, and if $\mu(E) < \infty$ for at least one set $E$, then (iv) is also satisfied.

\begin{solution}
  We have
  \begin{equation*}
    \mu(E) = \mu(E \cup \varnothing) = \mu(E) + \mu(\varnothing),
  \end{equation*}
  hence $\mu(\varnothing) = 0$.
\end{solution}

\subsubsection*{1.2.2}
Let $X$ be an infinite space. Let $\cal A$ be the class of all subsets of $X$. Define $\mu(E) = 0$ if $E$ is finite and $\mu(E) = \infty$ if $E$ is infinite. Then $\mu$ is finitely additive but not completely additive.

\begin{solution}
  Suppose $A, B \in \cal A$. Note that $A \cup B$ is finite if both $A$ and $B$ are finite, but infinite otherwise. Hence
  \begin{equation*}
    \mu(A \cup B) = 0 = \mu(A) + \mu(B)
  \end{equation*}
  in the former case, and
  \begin{equation*}
    \mu(A \cup B) = \infty = \mu(A) + \mu(B)
  \end{equation*}
  in the latter. This proves that $\mu$ is additive; \emph{finite} additivity follows by a simple induction argument.

  Let $(x_n)$ be a sequence of distinct points in $X$. Then $\bigcup_{n=1}^\infty \{x_n\}$ is an infinite set, so
  \begin{equation*}
    \mu \left( \bigcup_{n=1}^\infty \{x_n\} \right) = \infty,
  \end{equation*}
  but
  \begin{equation*}
    \sum_{n=1}^\infty \mu(\{x_n\}) = 0.
  \end{equation*}
  Thus $\mu$ is not completely additive.
\end{solution}

\subsubsection*{1.2.3}
If $\mu$ is a measure on a $\sigma$-algebra $\cal A$, and if $E$, $F$ are sets of $\cal A$, then
\begin{equation*}
  \mu(E) + \mu(F) = \mu(E \cup F) + \mu(E \cap F).
\end{equation*}

\begin{solution}
  If $\mu(F) = \infty$, then $\mu(E \cup F) = \infty$ by Theorem 1.2.1(i), and the given equality holds. If $\mu(F) < \infty$, then
  \begin{equation*}
    \begin{split}
      \mu(E \cup F) &= \mu[E \cup (F - (E \cap F))] \\
      &= \mu(E) + \mu[F - (E \cap F)] \\
      &= \mu(E) + \mu(F) - \mu(E \cap F),
    \end{split}
  \end{equation*}
  with the last equality following from Theorem 1.2.1(ii). Note that $E \cap F \subset F$ so that $\mu(E \cap F) \le \mu(F) < \infty$. Hence we can rearrange the above to yield
  \begin{equation*}
    \mu(E) + \mu(F) = \mu(E \cup F) + \mu(E \cap F).
  \end{equation*}
\end{solution}

\subsubsection*{1.2.6}
Give an example of a measure $\mu$ and a monotone-decreasing sequence $\{E_n\}$ of $\cal A$ such that $\mu(E_n) = \infty$ for all $n$, and $\mu(\lim_n E_n) = 0$.

\begin{solution}
  Let $X = \bb R$ and let $\cal A = \cal P(\bb R)$ (the power set of $\bb R$; this is easily seen to be a $\sigma$-algebra). Define $\mu: \cal A \to [0,\infty]$ such that $\mu(E)$ is the number of points in $E$ (with $\mu(E) = \infty$ if $E$ is infinite). This is easily seen to be a measure.

  For each $n \in \bb N$, let $E_n = (0, 1/n)$. Then $(E_n)$ is a monotone decreasing sequence of sets in $\cal A$, $\mu(E_n) = \infty$ for all $n$, and
  \begin{equation*}
    \mu \left( \lim_n E_n \right) = \mu \left( \bigcap_{n=1}^\infty E_n \right) = \mu(\varnothing) = 0.
  \end{equation*}
\end{solution}

\section*{Section 1.3 -- Outer measure}

\subsection*{Problems}

\subsubsection*{1.3.1}
Define $\mu^*(E)$ as the number of points in $E$ if $E$ is finite and $\mu^*(E) = \infty$ if $E$ is infinite. Show that $\mu^*$ is an outer measure. Determine the measurable sets.

\begin{solution}
  Of the properties listed in Definition 1.3.1, only countable subadditivity is non-obvious for $\mu^*$. But let us start with proving finite subadditivity.

  Let $A$ and $B$ be sets. If either is infinite, then so is $A \cup B$, hence
  \begin{equation*}
    \mu^*(A \cup B) = \infty = \mu^*(A) + \mu^*(B).
  \end{equation*}
  If both $A$ and $B$ are finite sets, then
  \begin{equation*}
    \mu^*(A \cup B) = \mu^*(A) + \mu^*(B - A) \le \mu^*(A) + \mu^*(B)
  \end{equation*}
  by basic set-theoretic considerations. Thus $\mu^*$ is subadditive, and finite subadditivity follows by induction on the number of sets in the union.

  Now, let $(E_n)$ be a sequence of sets. If infinitely many of the sets $E_n$ are nonempty, then $\sum_n \mu^*(E_n) = \infty$, and
  \begin{equation*}
    \mu^*\left( \bigcup_n E_n \right) \le \sum_n \mu^*(E_n)
  \end{equation*}
  follows. If only finitely many of the sets $E_n$ are nonempty, let $E_{n_1}, E_{n_2}, \dots, E_{n_k}$ be those sets. Then
  \begin{equation*}
    \mu^*\left( \bigcup_{n=1}^\infty E_n \right) = \mu^*\left( \bigcup_{i=1}^k E_{n_i} \right) \le \sum_{i=1}^k \mu^*(E_{n_i}) = \sum_{n=1}^\infty \mu^*(E_n),
  \end{equation*}
  by finite subadditivity. This proves that $\mu^*$ is countably subadditive, and hence that $\mu^*$ is an outer measure.

  Note that $\mu^*$ is \emph{additive} on disjoint sets; if $A \cap B = \varnothing$, then $\mu^*(A \cup B) = \mu^*(A) + \mu^*(B)$. In particular,
  \begin{equation*}
    \mu^*(A) = \mu^*(A \cap E) + \mu^*(A - E)
  \end{equation*}
  for all sets $A, E$. That is, all sets are measurable.
\end{solution}

\subsubsection*{1.3.2}
Define $\mu^*(\varnothing) = 0$, $\mu^*(E) = 1$ if $E \ne \varnothing$. Show that $\mu^*$ is an outer measure, and determine the measurable sets.

\begin{solution}
  As in the previous excercise, the only slightly non-obvious property is countable subadditivity. Hence, let $(E_n)$ be a sequence of sets. If all the sets $E_n$ are empty, then certainly
  \begin{equation*}
    \mu^* \left( \bigcup_n E_n \right) = 0 = \sum_{n} \mu^*(E_n).
  \end{equation*}
  If not, then there is some $m$ such that $E_m \ne \varnothing$, and it follows that
  \begin{equation*}
    \mu^* \left( \bigcup_n E_n \right) = 1 = \mu^*(E_m) \le \sum_n \mu^*(E_n).
  \end{equation*}
  Thus $\mu^*$ is indeed countably subadditive, and therefore also an outer measure.

  The empty set is measurable:
  \begin{equation*}
    \mu^*(A \cap \varnothing) + \mu^*(A - \varnothing) = \mu^*(\varnothing) + \mu^*(A) = \mu^*(A)
  \end{equation*}
  for all sets $A$. It follows by Theorem 1.3.1 that $X$ is measurable as well (the measurable sets make up a $\sigma$-algebra). Indeed $\varnothing$ and $X$ are the only measurable sets. To see this, let $E$ be any set other than those two (this requires that $X$ contains at least two elements). Then both $E$ and $E^c$ are nonempty, so
  \begin{equation*}
    \mu^*(X \cap E) + \mu^*(X - E) = \mu^*(E) + \mu^*(E^c) = 2 > 1 = \mu^*(X).
  \end{equation*}
\end{solution}


\section*{Section 1.4 -- Construction of outer measures}

\subsection*{Problems}

\subsubsection*{1.4.4}
If $\sc K$ is a $\sigma$-algebra and $\lambda$ is a measure on $\sc K$, then $\mu^*(A) = \lambda(A)$ for any $A \in \sc K$. [\emph{Hint:} $\mu^*(A) = \inf \{\lambda(E); E \in \sc K, E \supset A\}$.]

\begin{solution}
  Note that the description of $\mu^*$ can be simplified when $\sc K$ is a $\sigma$-algebra and $\lambda$ is a measure. For suppose that $A \subset X$, $E_n \in \sc K \; (n = 1, 2, \dots$), and $A \subset \bigcup_n E_n$. Then $E := \bigcup_n E_n \in \sc K$, and $\lambda(E) \le \sum_n \lambda(E_n)$ by Theorem 1.2.2. Hence
  \begin{equation*}
    \mu^*(A) = \inf \{\lambda(E); E \in \sc K, E \supset A\}.
  \end{equation*}

  Now, suppose that $A \in \sc K$. Certainly $\lambda(A)$ is an element of $\{\lambda(E); E \in \sc K, E \supset A\}$. And if $E \in \sc K$ and $E \supset A$, then $\lambda(E) \ge \lambda(A)$ by Theorem 1.2.1(i). Thus
  \begin{equation*}
    \lambda(A) = \inf \{\lambda(E); E \in \sc K, E \supset A\} = \mu^*(A).
  \end{equation*}
\end{solution}

\subsubsection*{1.4.5}
If $\sc K$ is a $\sigma$-algebra and $\lambda$ is a measure on $\sc K$, then every set in $\sc K$ is $\mu^*$-measurable.

\begin{solution}
  Recall the simplified description of $\mu^*$ from the previous problem. Let $E \in \sc K$ and $A \subset X$. For every $\epsilon > 0$ there exists $F \in \sc K$ such that $F \supset A$ and
  \begin{equation*}
    \mu^*(A) + \epsilon > \lambda(F);
  \end{equation*}
  else $\mu^*(A)$ would not be the greatest lower bound of $\{\lambda(E); E \in \sc K, E \supset A\}$. Moreover,
  \begin{equation*}
    \lambda(F) = \lambda(F \cap E) + \lambda(F - E)
  \end{equation*}
  since $\lambda$ is a measure on $\sc K$,
  \begin{equation*}
    \lambda(F \cap E) + \lambda(F - E) = \mu^*(F \cap E) + \mu^*(F - E) 
  \end{equation*}
  by what we found in the previous exercise, and finally
  \begin{equation*}
    \mu^*(F \cap E) + \mu^*(F - E) \ge \mu^*(A \cap E) + \mu^*(A - E)
  \end{equation*}
  by monotonicity of the outer measure $\mu^*$. Putting all of this together, we have
  \begin{equation*}
    \mu^*(A) + \epsilon > \mu^*(A \cap E) + \mu^*(A - E)
  \end{equation*}
  for all $\epsilon > 0$, and thus
  \begin{equation*}
    \mu^*(A) \ge \mu^*(A \cap E) + \mu^*(A - E).
  \end{equation*}
  It follows that every set $E \in \sc K$ is $\mu^*$-measurable.
\end{solution}


\section*{Section 1.6 -- The Lebesgue and the Lebesgue-Stieltjes measures}

\subsection*{Problems}

\subsubsection*{1.6.3}
The outer Lebesgue measure of a closed bounded interval $[a,b]$ on the real line is equal to $b-a$. [\emph{Hint:} Use the Heine-Borel theorem to replace a countable covering by a finite covering.]

\begin{solution}
  Suppose $(E_n)$ is a sequence of elements of $\sc K$ (i.e.\ a sequence of open intervals) such that $[a,b] \subset \bigcup_{n=1}^\infty E_n$. The collection $\{E_n\}$ constitutes an \emph{open cover} of $[a,b]$. By the Heine-Borel theorem $[a,b]$ is compact, hence there exists a \emph{finite subcover} $\{E_{n_1}, \dots, E_{n_k}\}$, such that $[a,b] \subset \bigcup_{i=1}^k E_{n_i}$.

  Assume without loss of generality that $E_{n_i} \cap [a,b] \ne \varnothing$ for all $i$; otherwise we can simply remove those $E_{n_i}$ that are disjoint with $[a,b]$ and still have a finite subcover. Write $E_{n_i} = (a_i, b_i)$ for each $i$, and define
  \begin{equation*}
    \alpha = \min\{a_1, \dots, a_k\}, \quad \beta = \max\{b_1, \dots, b_k\}.
  \end{equation*}
  It is clear that $\alpha$ and $\beta$ are the infimum and supremum, respectively, of $\bigcup_{i=1}^k E_{n_i}$. Note that $\alpha = a_j$ for some $j$, and $a_j < a < b_j$ since $E_{n_j}$ and $[a,b]$ have nonempty intersection. Thus $(\alpha, a] \subset \bigcup_{i=1}^k E_{n_i}$, and similarly $[b, \beta) \subset \bigcup_{i=1}^k E_{n_i}$. It follows that
  \begin{equation*}
    \bigcup_{i=1}^k E_{n_i} = (\alpha, \beta) \in \sc K.
  \end{equation*}

  Finally note that $\lambda$ is finitely subadditive. (This is easily proven with induction.) Thus,
  \begin{equation*}
    \sum_{n=1}^\infty \lambda(E_n) \ge \sum_{i=1}^k \lambda(E_{n_i}) \ge \lambda\left[ (\alpha, \beta) \right] = \beta - \alpha > b - a.
  \end{equation*}
  It follows that $b - a$ is a lower bound of the set
  \begin{equation*}
    \Lambda([a,b]) := \left\{ \sum_{n=1}^\infty \lambda(E_n); \, E_n \in \sc K, \, \bigcup_{n=1}^\infty E_n \supset [a,b] \right\}.
  \end{equation*}
  Moreover, for every $\epsilon > 0$ we have
  \begin{equation*}
    [a,b] \subset \left( a - \frac{\epsilon}{2}, b + \frac{\epsilon}{2} \right) \in \sc K
  \end{equation*}
  and
  \begin{equation*}
    \lambda \left[ \left( a - \frac{\epsilon}{2}, b + \frac{\epsilon}{2} \right) \right] = b - a + \epsilon.
  \end{equation*}
  Hence $b - a$ is the \emph{greatest} lower bound of $\Lambda([a,b])$, and $\mu^*([a,b]) = b - a$.
\end{solution}

\subsubsection*{1.6.4}
The outer Lebesgue measure of each of the intervals $(a,b), [a,b), (a,b]$ is equal to $b-a$.

\begin{solution}
  Recall that $\mu^*$ is monotone, on account of being an outer measure. Hence $\mu^*[(a,b)] \le \mu^*([a,b]) = b - a$, the latter equality being the result of the previous problem. Moreover, for all $\epsilon \in (0, b-a)$ we have
  \begin{equation*}
    \left( a + \frac{\epsilon}{2}, b - \frac{\epsilon}{2} \right) \subset (a,b),
  \end{equation*}
  so that
  \begin{equation*}
    \mu^*[(a,b)] \ge \mu^* \left[ \left( a + \frac{\epsilon}{2}, b - \frac{\epsilon}{2} \right) \right] = b - a + \epsilon.
  \end{equation*}
  Thus $\mu^*[(a,b)] \ge b - a$, and it follows that $\mu^*[(a,b)] = b - a$.

  The outer measures of $[a,b)$ and $(a,b]$ follow immediately by monotonicity:
  \begin{equation*}
    \mu^*[(a,b)] \le \mu^*([a,b)) \le \mu^*([a,b]),
  \end{equation*}
  so that $\mu^*([a,b)) = b - a$. Similarly for $(a,b]$.
\end{solution}

\subsubsection*{1.6.5}
Consider the transformation $T x = \alpha x + \beta$ from the real line onto itself, where $\alpha, \beta$ are real numbers and $\alpha \ne 0$. It maps sets $E$ onto sets $T(E)$. Denote by $\mu$ ($\mu^*$) the Lebesgue measure (outer measure) on the real line. Prove
\begin{enumerate}[label=(\alph*)]
  \item For any set $E$, $\mu^*(T(E)) = |\alpha| \mu^*(E)$.
  \item $E$ is Lebesgue-measurable if and only if $T(E)$ is Lebesgue-measureable.
  \item If $E$ is Lebesgue-measurable, then $\mu(T(E)) = |\alpha| \mu(E)$.
\end{enumerate}

\begin{solution}
  Let us start with a couple of simple observations:
  \begin{itemize}
    \item $T$ is bijective, with inverse given by
      \begin{equation*}
        T^{-1}(x) = \frac{x - \beta}{\alpha}.
      \end{equation*}
    \item Suppose $I = (a,b)$. Then
      \begin{equation*}
        T(I) = (\alpha a + \beta, \alpha b + \beta)
      \end{equation*}
      if $\alpha > 0$, and
      \begin{equation*}
        T(I) = (\beta b + \beta, \alpha a + \beta)
      \end{equation*}
      if $\alpha < 0$. Either way,
      \begin{equation*}
        \mu^*[T(I)] = |\alpha| (b - a) = |\alpha| \mu^*(I),
      \end{equation*}
      where we have used one of the results of the previous exercise. Similarly, $T^{-1}(I)$ is an open interval and
      \begin{equation*}
        \mu^*[T^{-1}(I)] = |\alpha|^{-1} \mu^*(I).
      \end{equation*}
      Of course, the latter two identities still hold if $I = \varnothing$. Hence they hold for all $I \in \sc K$.
  \end{itemize}
  Also, let us use the notation
  \begin{equation*}
    \Lambda(E) = \left\{ \sum_{n=1}^\infty \lambda(I_n); \, I_n \in \sc K, \, \bigcup_{n=1}^\infty I_n \supset E \right\}
  \end{equation*}
  for all $E \subset \bb R$.

  \begin{enumerate}[label=(\alph*)]
    \item Suppose $(I_n)$ is a sequence in $\sc K$ (i.e.\ a sequence of open intervals) and $E \subset \bigcup_n I_n$. Then $T(I_n) \in \sc K$ for every $n$,
      \begin{equation*}
        T(E) \subset T \left( \bigcup_n I_n \right) = \bigcup_n T(I_n),
      \end{equation*}
      and
      \begin{equation*}
        \sum_n \lambda[T(I_n)] = |\alpha| \sum_n \lambda(I_n).
      \end{equation*}
      Thus, if $s \in \Lambda(E)$, then $|\alpha| s \in \Lambda[T(E)]$. It follows that
      \begin{equation*}
        \mu^*[T(E)] = \inf \Lambda[T(E)] \le |\alpha| \inf \Lambda(E) = |\alpha| \mu^*(E).
      \end{equation*}

      Conversely, suppose $(J_n)$ is a sequence in $\sc K$ and $T(E) \subset \bigcup_n J_n$. Then $T^{-1}(F_n) \in \sc K$ for all $n$,
      \begin{equation*}
        E = T^{-1}[T(E)] \subset T^{-1} \left( \bigcup_n J_n \right) = \bigcup_n T^{-1}(J_n),
      \end{equation*}
      and
      \begin{equation*}
        \sum_n \lambda[T^{-1}(J_n)] = |\alpha|^{-1} \sum_n \lambda(J_n).
      \end{equation*}
      Hence, by the same logic as above, we find that $\mu^*(E) \le |\alpha|^{-1} \mu^*[T(E)]$, and it follows that
      \begin{equation*}
        \mu^*[T(E)] = |\alpha| \mu^*(E).
      \end{equation*}

    \item Note that if $f: X \to Y$ is a bijective function (between arbitrary sets $X,Y$), then
      \begin{equation*}
        \begin{split}
          f^{-1}[f(A)] &= A, \\
          f(A \cup B) &= f(A) \cup f(B), \\
          f(A - B) &= f(A) - f(B), \\
          f[f^{-1}(C)] &= C, \\
        \end{split}
      \end{equation*}
      for all $A, B \subset X$ and $C \subset Y$.

      Suppose that $E$ is measureable:
      \begin{equation*}
        \mu^*(A) = \mu^*(A \cap E) + \mu^*(A - E)
      \end{equation*}
      for all $A \subset \bb R$. Then, for all $B \subset \bb R$, we have
      \begin{equation*}
        \begin{split}
          \mu^*[B \cap T(&E)] + \mu^*[B - T(E)] \\
          &= \mu^*[T(T^{-1}(B) \cap E)] + \mu^*[T(T^{-1}(B) - E)] \\
          &= |\alpha| \mu^*[T^{-1}(B) \cap E] + |\alpha| \mu^*[T^{-1}(B) - E] \\
          &= |\alpha| \mu^*[T^{-1}(B)] \\
          &= \mu^*(B),
        \end{split}
      \end{equation*}
      so that $T(E)$ is measurable.

      Conversely, suppose that $T(E)$ is measurable. Then, for all $A \subset \bb R$,
      \begin{equation*}
        \begin{split}
          \mu^*(A \cap E&) + \mu^*(A - E) \\
          &= \mu^*[T^{-1}(T(A) \cap T(E))] + \mu^*[T^{-1}(T(A) - T(E))] \\
          &= |\alpha|^{-1} \mu^*[T(A) \cap T(E)] + |\alpha|^{-1} \mu^*[T(A) - T(E)] \\
          &= |\alpha|^{-1} \mu^*[T(A)] \\
          &= \mu^*(A),
        \end{split}
      \end{equation*}
      so that $E$ is measurable.

    \item This is immediate given (a), (b), and the definition of the Lebesgue-measure. First, $T(E)$ is Lebesgue-measurable by (b). Next, $\mu(E) = \mu^*(E)$ and $\mu[T(E)] = \mu^*[T(E)]$ since $\mu$ is simply the restriction of $\mu^*$ to the measurable sets. Finally, $\mu^*[T(E)] = |\alpha| \mu^*(E)$ by (a).
  \end{enumerate}
\end{solution}

\end{document}
