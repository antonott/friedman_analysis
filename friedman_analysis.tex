\documentclass{report}
\usepackage[utf8]{inputenc}
\usepackage{amsmath}
\usepackage{amssymb}
\usepackage{amsthm}
\usepackage{mathrsfs}
\usepackage{enumitem}
\usepackage{array}

\newcommand{\bb}[1]{\mathbb{#1}}
\newcommand{\norm}[1]{\lVert #1 \rVert}
\newcommand{\brac}[1]{\langle #1 \rangle}

\let\sc\relax
\newcommand{\sc}[1]{\mathscr{#1}}

\let\cal\relax
\newcommand{\cal}[1]{\mathcal{#1}}

\theoremstyle{remark}
\newtheorem*{solution}{Solution}

% \setenumerate{listparindent=\parindent, parsep=0pt}

\title{Notes for \textit{Foundations of Modern Analysis} by Avner Friedman}
\author{Anton Ottosson -- antonott@kth.se}
\date{\today}

\begin{document}

\maketitle

\chapter*{Chapter 1 -- Measure Theory}

\section*{Section 1.1 -- Rings and algebras}

\subsection*{Problems}

\subsubsection*{1.1.1}
\begin{equation*}
  \left( \varliminf_n E_n \right)^c = \varlimsup_n E_n^c, \quad \left( \varlimsup_n E_n \right)^c = \varliminf_n E_n^c.
\end{equation*}

\begin{solution}
  For the first identity, note that
  \begin{equation*}
    \begin{split}
      x \in \left( \varliminf_n E_n \right)^c &\iff x \notin \varliminf_n E_n \\
      &\iff \text{$x \notin E_n$ for infinitely many $n$} \\
      &\iff \text{$x \in E_n^c$ for infinitely many $n$} \\
      &\iff x \in \varlimsup_n E_n^c.
    \end{split}
  \end{equation*}
  For the second,
  \begin{equation*}
    \begin{split}
      x \in \left( \varlimsup_n E_n \right)^c &\iff x \notin \varlimsup_n E_n \\
      &\iff \text{$x \in E_n$ for finitely many $n$} \\
      &\iff \text{$x \in E_n^c$ for all but finitely many $n$} \\
      &\iff x \in \varliminf_n E_n^c.
    \end{split}
  \end{equation*}
\end{solution}

\subsubsection{1.1.2}
\begin{equation*}
  \varlimsup_n E_n = \bigcap_{k=1}^\infty \bigcup_{n=k}^\infty E_n, \quad \varliminf_n E_n = \bigcup_{k=1}^\infty \bigcap_{n=k}^\infty E_n.
\end{equation*}

\begin{solution}
  Suppose $x \in \varlimsup_n E_n$. Then $x \in E_n$ for infinitely many $n$. It follows that $x \in \bigcup_{n=k}^\infty E_n$ for all $k \in \bb N$, and hence that $x \in \bigcap_{k=1}^\infty \bigcup_{n=k}^\infty E_n$.

  Conversely, assume that $x \in \bigcap_{k=1}^\infty \bigcup_{n=k}^\infty E_n$. Then $x \in \bigcup_{n=k}^\infty E_n$ for all $k \in \bb N$. It follows that $x \in E_n$ for infinitely many $n$, and thus that $x \in \varlimsup_n E_n$. This proves the first identity.

  Next, suppose that $x \in \varliminf_n E_n$. Then $x \in E_n$ for all but finitely many $n$, so there is some $k' \in \bb N$ such that $x \in E_n$ for all $n \ge k'$. It follows that $x \in \bigcap_{n=k'}^\infty E_n$, and hence that $x \in \bigcup_{k=1}^\infty \bigcap_{n=k}^\infty E_n$.

  Conversely, assume that $x \in \bigcup_{k=1}^\infty \bigcap_{n=k}^\infty E_n$. Then $x \in \bigcap_{n=k'}^\infty E_n$ for some $k' \in \bb N$, which means that $x \in E_n$ for all $n \ge k'$. It follows that $x \in E_n$ for all but finitely many $n$; that is, $x \in \varliminf_n E_n$.
\end{solution}

\subsubsection*{1.1.3}
If $\sc R$ is a $\sigma$-ring and $E_n \in \sc R$, then
\begin{equation*}
  \bigcap_{n=1}^\infty E_n \in \sc R, \quad \varlimsup_n E_n \in \sc R, \quad \varliminf_n E_n \in \sc R.
\end{equation*}

\begin{solution}
  Let $Y = \bigcup_{n=1}^\infty E_n$. Then $E_n \subset Y$ for all $Y$, and it follows that
  \begin{equation*}
    \bigcap_{n=1}^\infty E_n = Y \cap \left( \bigcap_{n=1}^\infty E_n \right) = Y - \left( Y - \bigcap_{n=1}^\infty E_n \right).
  \end{equation*}
  Notice that
  \begin{equation*}
    Y - \bigcap_{n=1}^\infty E_n = \bigcup_{n=1}^\infty (Y - E_n) \in \sc R,
  \end{equation*}
  by properties (b) and (e). (The equality is analogous to the identity (1.1.2), but with $Y$ in place of $X$.) Hence, applying (b) again, we find that
  \begin{equation*}
    \bigcap_{n=1}^\infty E_n = Y - \left( Y - \bigcap_{n=1}^\infty E_n \right) \in \sc R.
  \end{equation*}
  For later reference, let us call this result (x).

  Given $k \in \bb N$, let $A_n = \varnothing$ for $n < k$, and let $A_n = E_n$ for $n \ge k$. Then $A_n \in \sc R$ for all $n$ by (a), hence
  \begin{equation*}
    \bigcup_{n=k}^\infty E_n = \bigcup_{n=1}^\infty A_n \in \sc R
  \end{equation*}
  by (e). It then follows by (x) that
  \begin{equation*}
    \varlimsup_n E_n = \bigcap_{k=1}^\infty \bigcup_{n=k}^\infty E_n \in \sc R.
  \end{equation*}

  By a similar argument we find that (x) implies
  \begin{equation*}
    \bigcap_{n=k}^\infty E_n \in \sc R
  \end{equation*}
  for all $k \in \bb N$. Hence
  \begin{equation*}
    \varliminf_n E_n = \bigcup_{k=1}^\infty \bigcap_{n=k}^\infty E_n \in \sc R
  \end{equation*}
  by (e).
\end{solution}

\subsubsection*{1.1.4}
The intersection of any collection of rings (algebras, $\sigma$-rings, or $\sigma$-algebras) is also a ring (an algebra, $\sigma$-ring, or $\sigma$-algebra).

\begin{solution}
  Let $\sc C$ be a collection of classes. Let $\bigcap \sc C$ denote the intersection of all classes in $\sc C$. We will show that if one of the properties (a)-(e) is satisfied by all classes in $\sc C$, then $\bigcap \sc C$ satisfies that property as well. The result requested in the problem then follows as an immediate corollary.

  It is clear that if every $\sc R \in \sc C$ satisfies (a), then so does $\bigcap \sc C$. Suppose every $\sc R \in \sc C$ satisfies (b). If $A,B \in \bigcap \sc C$ then $A,B \in \sc R$ for every $\sc R \in \sc C$. Hence $A - B \in \sc R$ for all $\sc R \in \sc C$, and it follows that $A - B \in \bigcap \sc C$. The argument for (c) is similar (with $A \cup B$ in place of $A - B$), and (d) is obvious.

  Finally, suppose that every $\sc R \in \sc C$ satisfies (e). If $A_1, A_2, \dotsc \in \bigcap \sc C$ then $A_1, A_2, \dotsc \in \sc R$ for every $\sc R \in \sc C$. Hence $\bigcup_{n=1}^\infty A_n \in \sc R$ for all $\sc R \in \sc C$, and it follows that $\bigcup_{n=1}^\infty A_n \in \bigcap \sc C$.
\end{solution}

\subsubsection*{1.1.5}
If $\sc D$ is any class of sets, then there exists a unique ring $\sc R_0$ such that (i) $\sc R_0 \supset \sc D$, and (ii) any ring $\sc R$ containing $\sc D$ contains also $\sc R_0$. $\sc R_0$ is called the \emph{ring generated} by $\sc D$, and is denoted by $\sc R(\sc D)$.

\begin{solution}
  Let $\sc R_0$ be the intersection of all rings containing $\sc D$. This is a ring by the previous exercise, and it satisfies the properties (i) and (ii). To see that it is unique, let $\sc R_0'$ also by a ring satisfying (i) and (ii). Then $\sc R_0 \subset \sc R_0'$ and $\sc R_0' \subset \sc R_0$ by property (ii), so $\sc R_0 = \sc R_0'$.
\end{solution}

\subsubsection*{1.1.6}
If $\sc D$ is any class of sets, then there exists a unique $\sigma$-ring $\sc S_0$ such that (i) $\sc S_0 \supset \sc D$, and (ii) any $\sigma$-ring containing $\sc D$ contains also $\sc S_0$. We call $\sc S_0$ the \emph{$\sigma$-ring generated} by $\sc D$, and denote it by $\sc S(\sc D)$. A similar result holds for $\sigma$-algebras, and we speak of the \emph{$\sigma$-algebra generated} by $\sc D$.

\begin{solution}
  By the same argument as in the previous exercise, $\sc S_0$ is the intersection of all $\sigma$-rings containing $\sc D$. Similarly the $\sigma$-algebra generated by $\sc D$ is the intersection of all $\sigma$-algebras containing $\sc D$.
\end{solution}

\end{document}
