\documentclass{report}
\usepackage[utf8]{inputenc}
\usepackage{amsmath}
\usepackage{amssymb}
\usepackage{amsthm}
\usepackage{mathrsfs}
\usepackage{enumitem}
\usepackage{array}

\newcommand{\bb}[1]{\mathbb{#1}}
\newcommand{\norm}[1]{\lVert #1 \rVert}
\newcommand{\brac}[1]{\langle #1 \rangle}

\let\sc\relax
\newcommand{\sc}[1]{\mathscr{#1}}

\let\cal\relax
\newcommand{\cal}[1]{\mathcal{#1}}

\theoremstyle{remark}
\newtheorem*{solution}{Solution}

% \setenumerate{listparindent=\parindent, parsep=0pt}

\title{Notes for \textit{Foundations of Modern Analysis} by Avner Friedman}
\author{Anton Ottosson -- antonott@kth.se}
\date{\today}

\begin{document}

\maketitle

\chapter*{Chapter 1 -- Measure Theory}

\section*{Section 1.1 -- Rings and algebras}

\subsection*{Problems}

\subsubsection*{1.1.1}
\begin{equation*}
  \left( \varliminf_n E_n \right)^c = \varlimsup_n E_n^c, \quad \left( \varlimsup_n E_n \right)^c = \varliminf_n E_n^c.
\end{equation*}

\begin{solution}
  For the first identity, note that
  \begin{equation*}
    \begin{split}
      x \in \left( \varliminf_n E_n \right)^c &\iff x \notin \varliminf_n E_n \\
      &\iff \text{$x \notin E_n$ for infinitely many $n$} \\
      &\iff x \in \varlimsup_n E_n^c.
    \end{split}
  \end{equation*}
  For the second,
  \begin{equation*}
    \begin{split}
      x \in \left( \varlimsup_n E_n \right)^c &\iff x \notin \varlimsup_n E_n \\
      &\iff \text{$x \in E_n$ for finitely many $n$} \\
      &\iff \text{$x \in E_n^c$ for all but finitely many $n$} \\
      &\iff x \in \varliminf_n E_n^c.
    \end{split}
  \end{equation*}
\end{solution}

\subsubsection{1.1.2}
\begin{equation*}
  \varlimsup_n E_n = \bigcap_{k=1}^\infty \bigcup_{n=k}^\infty E_n, \quad \varliminf_n E_n = \bigcup_{k=1}^\infty \bigcap_{n=k}^\infty E_n.
\end{equation*}

\begin{solution}
  Suppose $x \in \varlimsup_n E_n$. Then $x \in E_n$ for infinitely many $n$. It follows that $x \in \bigcup_{n=k}^\infty E_n$ for all $k \in \bb N$, and hence that $x \in \bigcap_{k=1}^\infty \bigcup_{n=k}^\infty E_n$.

  Conversely, assume that $x \in \bigcap_{k=1}^\infty \bigcup_{n=k}^\infty E_n$. Then $x \in \bigcup_{n=k}^\infty E_n$ for all $k \in \bb N$. It follows that $x \in E_n$ for infinitely many $n$, and thus that $x \in \varlimsup_n E_n$. This proves the first identity.

  Next, suppose that $x \in \varliminf_n E_n$. Then $x \in E_n$ for all but finitely many $n$, so there is some $k' \in \bb N$ such that $x \in E_n$ for all $n \ge k'$. It follows that $x \in \bigcap_{n=k'}^\infty E_n$, and hence that $x \in \bigcup_{k=1}^\infty \bigcap_{n=k}^\infty E_n$.

  Conversely, assume that $x \in \bigcup_{k=1}^\infty \bigcap_{n=k}^\infty E_n$. Then $x \in \bigcap_{n=k'}^\infty E_n$ for some $k' \in \bb N$, which means that $x \in E_n$ for all $n \ge k'$. It follows that $x \in E_n$ for all but finitely many $n$; that is, $x \in \varliminf_n E_n$.
\end{solution}

\end{document}
